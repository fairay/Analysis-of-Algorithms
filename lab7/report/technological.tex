\section{Выбор языка программирования}
В качестве языка программирования был выбран C++\cite{C++_Doc}, так как имеется опыт работы с ним, и с библиотеками, позволяющими провести исследование и тестирование программы. Разработка проводилась в среде Visual Studio 2019\cite{VisualStudio}.

\section{Листинги кода}
Реализация алгоритмов поиска представлена на листингах 3.1-3.3. На листинге 3.4 представлена реализация разбиения словаря по сегментам

\begin{lstlisting}[caption = {Поиск полным перебором}]
rec_t full_search(const rec_arr& arr, size_t key)
{
	for (int i = 0; i < arr.size(); i++)
		if (arr[i].key == key)
			return arr[i];
	return null_rec();
}
\end{lstlisting}

\begin{lstlisting}[caption = {Поиск половинным разбиением}]
rec_t binary_search(const rec_arr& arr, size_t key)
{
	int left = 0;
	int right = arr.size() - 1;
	while (left <= right)
	{
		int mid = (left + right) / 2;
		if (key == arr[mid].key)
			return arr[mid];
		if (key < arr[mid].key)
			right = mid - 1;  
		else                  
			left = mid + 1;   
	}
	return null_rec();
}
\end{lstlisting}

\begin{lstlisting}[caption = {Поиск с сегментами}]
rec_t segment_search(const seg_arr& segments, size_t key)
{
	int seg_n = -1;
	for (int i=0; i<segments.size(); i++)
	if (segments[i].key == key % 10)
	{
		seg_n = i;
			break;
	}
	
	if (seg_n == -1)
		return null_rec();
	
	const rec_arr& arr = segments[seg_n].records;
	return full_search(arr, key);
}
\end{lstlisting}

\begin{lstlisting}[caption = {Разбиение словаря по сегментам}]
seg_arr split_arr(rec_arr& arr)
{
	seg_arr segments;
	for (int i = 0; i < 10; i++)
	{
		rec_seg temp_seg;
		temp_seg.key = i;
		
		segments.push_back(temp_seg);
	}
	
	for (int i = 0; i < arr.size(); i++)
		segments[arr[i].key % 10].records.push_back(arr[i]);
	return segments;
}
\end{lstlisting}


\section{Результаты тестирования}
Для тестирования написанных функций был создан отдельный файл с ранее описанными классами тестов. Тестирование функций проводилось за счёт сравнения результов функций друг с другом.

Состав тестов приведён в листинге 3.5.

\begin{lstlisting}[caption = {Модульные тесты}]
#include "tests.h"

using namespace std;

rec_t test1(rec_arr& arr, size_t key)
{
	return full_search(arr, key);
}
rec_t test2(rec_arr& arr, size_t key)
{
	sort_arr(arr);
	return binary_search(arr, key);
}
rec_t test3(rec_arr& arr, size_t key)
{
	seg_arr sarr = split_arr(arr);
	return segment_search(sarr, key);
}


bool _cmp_rec(const rec_t& r1, const rec_t& r2)
{
	return (r1.key == r2.key) && (r1.val == r2.val);
}

bool _test_all(rec_arr& arr, size_t key, rec_t res)
{
	test_f test_f_arr[3] = { test1, test2, test3 };
	rec_t test_out;
	
	for (int i = 0; i < 3; i++)
	{
		test_out = test_f_arr[i](arr, key);
		if (!_cmp_rec(test_out, res))
		return false;
	}
	return true;
}

void _find_missing(rec_arr& arr)
{
	if (_test_all(arr, 1012, null_rec()))
	cout << __FUNCTION__ << " - OK\n";
	else
	cout << __FUNCTION__ << " - FAILED\n";
}
void _find_first(rec_arr& arr)
{
	if (_test_all(arr, arr[0].key, arr[0]))
	cout << __FUNCTION__ << " - OK\n";
	else
	cout << __FUNCTION__ << " - FAILED\n";
}
void _find_last(rec_arr& arr)
{
	size_t last = arr.size() - 1;
	if (_test_all(arr, arr[last].key, arr[last]))
	cout << __FUNCTION__ << " - OK\n";
	else
	cout << __FUNCTION__ << " - FAILED\n";
}

void run_tests(rec_arr& arr)
{
	cout << "Running tests:" << endl;
	_find_missing(arr);
	_find_first(arr);
	_find_last(arr);
	cout << endl;
}
\end{lstlisting}


\section{Оценка времени}
Для замера процессорного времени исполнения функции используется функция QueryPerformanceCounter библиотеки windows.h\cite{QueryPerformanceCounter}. Код функций замера времени приведёны в листинге 3.6.

\begin{lstlisting}[caption = {Функции замера процессорного времени работы функции}]
double PCFreq = 0.0;
__int64 CounterStart = 0;

void start_counter()
{
	LARGE_INTEGER li;
	QueryPerformanceFrequency(&li);
	
	PCFreq = double(li.QuadPart) / 1000.0;
	
	QueryPerformanceCounter(&li);
	CounterStart = li.QuadPart;
}

double get_counter()
{
	LARGE_INTEGER li;
	QueryPerformanceCounter(&li);
	return double(li.QuadPart - CounterStart) / PCFreq;
}
\end{lstlisting}

\section*{Вывод}
Результатом технологической части стал выбор используемых технических средств реализации и реализация алгоритмов, системы тестов и замера времени работы на языке С++.
