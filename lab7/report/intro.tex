В данной лабораторной реализуются и оцениваются различные алгоритмы поиска ключей в словаре банковских карт.

Словари используются для связывания двух понятий, одно из которых называется ключом, а другое значением. Сам словарь хранит множество таких пар и предоставляет определённые функции для работы с хранимыми значениями. Одной из самых часто используемых операций в таком случае является операция получения значения по заданному ключу. В случае, когда словарь имеет большое количество записей, например более одного миллиона, задача уменьшения времени поиска ключей становится особенно актуальной. Поэтому существует множество алгоритмов, осуществляющих эту задачу.

В данной лабораторной работе в качестве примера подобных алгоритмов будут рассмотрены:
\begin{itemize}
	\item поиск полным перебором;
	\item поиск половинным делением;
	\item поиск полным перебором с использованием сегментации.
\end{itemize}
