\section{Цель и задачи работы}
Целью лабораторной работы является разработка и исследование алгоритма поиска ключей в словаре банковских карт.

Выделены следующие задачи лабораторной работы:

\begin{itemize}
\item описание понятия словаря;
\item описание и реализация алгоритмов поиска ключей в словаре;
\item проведение замеров времени поиска ключей разными алгоритмами.
\end{itemize}

\section{Описание понятия словаря}
Словарь - массив, состоящий из пар вида <<(ключ, значение)>>, предоставляющий возможность вставки нового элемента, удаления и поиска по ключу\cite{Corman}. В качестве подобного словаря для этой лабораторной работы был выбран словарь банковских карт. В качестве ключа выступает номер карты, а в качестве значения CVC ключ.


\section{Используемые алгоритмы}
В рамках лабораторной работы было поставлено написание трёх алгоритмов поиска ключа в словаре.

\subsection{Поиск полным перебором}
	Данный алгоритм проверяет все ключи на признак совпадения с искомым до нахождения его в словаре или до исчерпания возможных вариантов.

\subsection{Поиск половинным делением}
	Алгоритм работает с отсортированным массивом ключей. Производится выбор среднего элемента массива и сравнение его с искомым ключом. На основе результата сравнения принимается решение, какую из половин массива (правее или левее середины) следует взять для дальнейшего повторения данной операции. Такой алгоритм имеет меньшее среднее время работы, что достигается за счёт того, что для массива размером $N$ потребуется сделать не более $\log_{2}(N)$ сравнений.
	
\subsection{Поиск с сегментацией}
	Другим способом снижения среднего времени поиска является разбиение массива на сегменты по некоторым схожим признакам ключей, например, одинаковые первые символы. Также можно применить частотный анализ для определения количества запросов к выделенным сегментам и на основе этого сконфигурировать их таким образом, чтобы время поиска ключей было оптимизировано под нужды конкретной задачи.
	
	Таким образом, осуществление поиска осуществляется в два этапа: поиск сегмента, под правило которого подходит ключ, поиск ключа среди элементов сегмента.

\section*{Вывод}
Результатом аналитического раздела стало определение цели и задач работы, описано понятие словаря, используемых алгоритмов поиска.