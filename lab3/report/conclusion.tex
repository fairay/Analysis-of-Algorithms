В ходе лабораторной работы достигнута поставленная цель: оценка трудоёмкости алгоритмов сортировки массиов. Решены все задачи работы.

Были изучены и описаны понятия трудоёмкости и операции сортировки. Также были описаны, реализованы и оценены по трудоёмкости алгоритмы сортировки массивов. Проведены замеры процессорного времени работы каждого алгоритма при различных размерах массивов. На основании оценок и экспериментов проведён сравнительный анализ.

Из расчётов трудоёмкости и проведённых экспериментов было выявлено, что наибольшим приростом процессорного времени на случайно сгенерированном массиве целых чисел обладает сортировка пузырьком. Также было выявлено, что поразрядная сортировка обладает меньшим приростом в процессорном времени при увеличении размерности массива по сравнению с сортировкой слиянием даже при наличии в массиве отрицательных чисел, что является худшим случаем для поразрядной сортировки.