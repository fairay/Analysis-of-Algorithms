В ходе лабораторной работы достигнута поставленная цель: оценка трудоёмкости алгоритма умножения матриц и получение
практического навыка оптимизации алгоритмов. Решены все задачи работы.

Были изучены и описаны понятия трудоёмкости и операции умножения матриц. Также были описаны и реализованы алгоритмы умножения матриц. Был оптимизирован алгоритм Винограда. Проведены замеры процессорного времени работы каждого алгоритма при различных размерах матриц (в том числе чётных и нечётных), оценена трудоёмкость. На основании оценок и экспериментов проведён сравнительный анализ.