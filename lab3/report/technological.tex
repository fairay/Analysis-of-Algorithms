\section{Выбор языка программирования}
В качестве языка программирования был выбран C++, так как имеется опыт работы с ним, и с библиотеками, позволяющими провести исследование и тестирование программы. Также в языке имеются средства для отключения оптимизации компилятора.


\section{Листинг кода}
Реализация алгоритмов умножения матриц представлена на листингах 3.1-3.3.

\begin{lstlisting}[caption = Функция умножения матриц классическим алгоритмом.]
#include "classic.h"
#pragma optimize( "", off )
mat_t classic_mult(mat_t a, mat_t b, int m, int n, int q)
{
	mat_t c = create_mat(m, q);
	
	for (int i = 0; i < m; i++)
	for (int j = 0; j < q; j++)
	{
		c[i][j] = 0;
		for (int k = 0; k < n; k++)
		c[i][j] += a[i][k] * b[k][j];
	}
	return c;
}
#pragma optimize( "", on )
\end{lstlisting}

\begin{lstlisting}[caption = Функция умножения матриц алгоритмом Винограда.]
#include "winograd.h"

#pragma optimize( "", off )

arr_t calc_mi(mat_t a, int m, int n)
{
	arr_t mi = create_arr(m);
	for (int i = 0; i < m; i++)
	{
		mi[i] = 0;
		for (int k = 0; k < n / 2; k++)
			mi[i] += a[i][2*k] * a[i][2*k + 1];
	}
	return mi;
}
arr_t calc_mj(mat_t b, int n, int q)
{
	arr_t mj = create_arr(q);
	for (int j = 0; j < q; j++)
	{
		mj[j] = 0;
		for (int k = 0; k < n / 2; k++)
			mj[j] += b[2*k][j] * b[2*k + 1][j];
	}
	return mj;
}
mat_t winograd_mult(mat_t a, mat_t b, int m, int n, int q)
{
	mat_t c = create_mat(m, q);
	arr_t mi = calc_mi(a, m, n);
	arr_t mj = calc_mj(b, n, q);
	for (int i = 0; i < m; i++)
		for (int j = 0; j < q; j++)
		{
			c[i][j] = -(mi[i] + mj[j]);
			for (int k = 0; k < n / 2; k++)
				c[i][j] +=	(a[i][2*k] + b[2*k + 1][j]) * 
							(a[i][2*k + 1] + b[2*k][j]);
		}
	if (n % 2)
		for (int i = 0; i < m; i++)
			for (int j = 0; j < q; j++)
				c[i][j] += a[i][n-1] * b[n-1][j];
	return c;
}

#pragma optimize( "", on )
\end{lstlisting}

\begin{lstlisting}[caption = Оптимизированая функция умножения матриц алгоритмом Винограда.]
#include "winograd.h"

#pragma optimize( "", off )

arr_t calc_mj(mat_t b, int n, int q)
{
	arr_t mj = create_arr(q);
	for (int j = 0; j < q; j++)
	{
		double mjj = 0; 
		for (int k = 1; k < n; k += 2)
			mjj += b[k][j] * b[k - 1][j];
		mj[j] = mjj;
	}
	return mj;
}

mat_t winograd_mult(mat_t a, mat_t b, int m, int n, int q)
{
	mat_t c = create_mat(m, q);
	arr_t mj = calc_mj(b, n, q);
	
	for (int i = 0; i < m; i++)
	{
		double mi_i = 0;
		for (int k = 1; k < n; k += 2)
			mi_i += a[i][k] * a[i][k - 1];
		
		for (int j = 0; j < q; j++)
		{
			double cij = -(mi_i + mj[j]);
			int k = 1;
			int k1 = 0;
			for (; k < n; k += 2, k1 += 2)
				cij +=	(a[i][k] + b[k1][j]) * (a[i][k1] + b[k][j]);
			c[i][j] = cij;
		}
	}
	
	if (n % 2)
	{
		int n_minus1 = n - 1;
		for (int i = 0; i < m; i++)
			for (int j = 0; j < q; j++)
				c[i][j] += a[i][n_minus1] * b[n_minus1][j];
	}
	
	free_arr(&mj);
	return c;
}

#pragma optimize( "", on )
\end{lstlisting}


\section{Результаты тестирования}
Для тестирования написанных функций был создан отдельный файл с вышеописаными классами тестов. Тестирование функций проводилось за счёт сравнения результов двух функций.

Состав тестов приведён в листинге 3.4.

\begin{lstlisting}[caption = Модульные тесты]
#include "tests.h"
// Сравнение результата умножения разными способами
bool _cmp_funcs(mat_t a, mat_t b, int m, int n, int q)
{
	mat_t c1 = classic_mult(a, b, m, n, q);
	mat_t c2 = winograd_mult(a, b, m, n, q);
	bool flag = cmp_matrix(c1, c2, m, q);
	free_mat(&c1, m, q);
	free_mat(&c2, m, q);
	return flag;
}

// Матрицы с размером 1x1
void _size_one_test()
{
	mat_t a = create_mat(1, 1);
	mat_t b = create_mat(1, 1);
	
	a[0][0] = 0;
	b[0][0] = 1;
	if (!_cmp_funcs(a, b, 1, 1, 1))
	{
		std::cout << __FUNCTION__ << " - FAILED\n";
		return;
	}
	
	a[0][0] = 3;
	b[0][0] = 4;
	if (!_cmp_funcs(a, b, 1, 1, 1))
	{
		std::cout << __FUNCTION__ << " - FAILED\n";
		return;
	}
	
	free_mat(&a, 1, 1);
	free_mat(&b, 1, 1);
	
	std::cout << __FUNCTION__ << " - OK\n";
}
// Нулевые матрицы
void _void_test()
{
	mat_t a = random_matrix(3, 2);
	mat_t b = void_matrix(2, 1);
	if (!_cmp_funcs(a, b, 3, 2, 1))
	{
		std::cout << __FUNCTION__ << " - FAILED\n";
		return;
	}
	free_mat(&a, 3, 2);
	a = void_matrix(3, 2);
	if (!_cmp_funcs(a, b, 3, 2, 1))
	{
		std::cout << __FUNCTION__ << " - FAILED\n";
		return;
	}
	free_mat(&a, 3, 2);
	free_mat(&b, 2, 1);
	std::cout << __FUNCTION__ << " - OK\n";
}
// Квадратные матрицы
void _square_test()
{
	mat_t a = random_matrix(4, 4);
	mat_t b = random_matrix(4, 4);
	
	if (!_cmp_funcs(a, b, 4, 4, 4))
	{
		std::cout << __FUNCTION__ << " - FAILED\n";
		return;
	}
	
	free_mat(&a, 4, 4);
	free_mat(&b, 4, 4);
	std::cout << __FUNCTION__ << " - OK\n";
}
// Матрицы нечётного размера
void _odd_test()
{
	mat_t a = random_matrix(5, 3);
	mat_t b = random_matrix(3, 7);
	
	if (!_cmp_funcs(a, b, 5, 3, 7))
	{
		std::cout << __FUNCTION__ << " - FAILED\n";
		return;
	}
	
	free_mat(&a, 5, 3);
	free_mat(&b, 3, 7);
	std::cout << __FUNCTION__ << " - OK\n";
}

void run_tests()
{
	_size_one_test();
	_void_test();
	_square_test();
	_odd_test();
}
\end{lstlisting}

Все тесты пройдены успешно.


\section{Оценка трудоёмкости}
Произведём оценку трудоёмкости алгоритов. Будем считать, что сортируется массив A$[N]$\\, максимальная разница двух элементов состоит из $K$ элементов.

\subsection{Алгоритм сортировки пузырьком}
	\par $ f_{bub} = 2 + (N-1)\cdot(2 + 3 + \frac{N-1 + 1}{2}\cdot[3 + 4 + 
	\left\{\begin{array}{ll}
		0, & $л.с.$\\
		9, & $х.с.$
	\end{array}\right.  ])$
	
	\par $ f_{bub} = 2 + (N-1)\cdot(5 + \frac{N}{2}\cdot[7 + 
	\left\{\begin{array}{ll}
		0, & $л.с.$\\
		9, & $х.с.$
	\end{array}\right.  ])$
	
	Лучший случай (отсортированный массив):
	\par $ f_{bub} = 2 + (N-1)\cdot(5 + \frac{N}{2}\cdot 7) = \frac{7}{2}N^{2} + \frac{3}{2}N - 3 $
	
	Худший случай (массив в обратном порядке):
	\par $ f_{bub} = 2 + (N-1)\cdot(5 + 9N) = 9N^{2} - 4N - 3$

\subsection{Алгоритм поразрядной сортировки} 
\textbf{Функция нормализации массива:}
	\par $ f_{format} = 2 + 2 + N\cdot[2 + 4 + 
		\left\{\begin{array}{ll}
			0, & $л.с.$\\
			1, & $х.с.$
		\end{array}\right.] 
	+ 2 + 
	\left\{\begin{array}{ll}
		0, & $л.с.$\\
		1 + 2 + N\cdot(2 + 2), & $х.с.$
	\end{array}\right. + 1 + (2 + 3K) $

	\par $ f_{format} = 9 + 3K + 7N +  
	\left\{\begin{array}{ll}
		0, & $л.с.$\\
		3 + 4N, & $х.с.$
	\end{array}\right. $

\textbf{Функция получения значения разряда:}
	\par $ f_{dig} = 3 + 2 = 5 $

\textbf{Функция поразрядной сортировки:}	
	\par $ f_{rad} = f_{format} + 1 + 2 +
	K\cdot(2 + 2 + 10(2 + 1) + 2 +
		N\cdot(2 + f_{dig} + 2) + 1 + 2 +
		10(2 + 6) + 2 + N\cdot(2 + f_{dig} + 5) + 3) + 1 +
	\left\{\begin{array}{ll}
		0, & $л.с.$\\
		2 + N\cdot(2 + 2), & $х.с.$
	\end{array}\right. $

	\par $ f_{rad} = 24KN + 125K + 7N + 12 + 
	\left\{\begin{array}{ll}
		0, & $л.с.$\\
		5 + 8N, & $х.с.$
	\end{array}\right.$
	
	Лучший случай (без отрицательных чисел):
	\par $ f_{rad} = 24KN + 125K + 7N + 12$
	
	Худший случай (с отрицательными числами):
	\par $ f_{rad} = 24KN + 125K + 15N + 17$
	

\subsection{Алгоритм сортировки слиянием} 
	\par $ f_{mer} = 2 + \frac{N}{2}\cdot(2 + 4 + 
	\left\{\begin{array}{ll}
		0, & $л.с.$\\
		10, & $х.с.$
	\end{array}\right.) + 1 + 2 + 
	\log_{2}(N)\cdot(2 + 2 + \dfrac{2(1-2^{\log_{2}(N)})}{(1 - 2)\cdot\log_{2}(N)}\cdot
	[4 + 2 + 1 + 
		\left\{\begin{array}{ll}
			1, & $л.с.$\\
			2, & $х.с.$
		\end{array}\right. +
		3 + 2 + 1 + \dfrac{log_{2}(N)*N}{\dfrac{2(1-2^{\log_{2}(N)})}{(1 - 2)}}\cdot
			\left\{\begin{array}{ll}
				(3 + 4 + 1)/2 + (3 + 3)/2, & $л.с.$\\
				3 + 4 + 1, & $х.с.$
			\end{array}\right.
	] + 5 + 3 + 
		\left\{\begin{array}{ll}
			0, & $л.с.$\\
			2 + \dfrac{2(1-2^{\log_{2}(N)})}{(1 - 2)\cdot\log_{2}(N)}\cdot(2 + 3), & $х.с.$
		\end{array}\right.
	)$
	\\
	
	\par $ f_{mer} = -21 + 29N + 
	\left\{\begin{array}{ll}
		0, & $л.с.$\\
		5N, & $х.с.$
	\end{array}\right. + 
	12\log_{2}(N) + 
		\left\{\begin{array}{ll}
			2N-2, & $л.с.$\\
			4N-4, & $х.с.$
		\end{array}\right. +
	(log_{2}(N)*N)\cdot
		\left\{\begin{array}{ll}
			7, & $л.с.$\\
			8, & $х.с.$
		\end{array}\right.
	 + 
	\left\{\begin{array}{ll}
		0, & $л.с.$\\
		2\log_{2}(N) + 10(N-1), & $х.с.$
	\end{array}\right.
	$ \\
	
	Лучший случай (N является степенью 2, отсортированный массив):
	\par $ f_{mer} = 7N\log_{2}(N) + 31N + 12\log_{2}(N) - 23 $
	
	Худший случай (N является степенью 2 - 1, случайный массив):
	\par $ f_{mer} = 8N\log_{2}(N) + 48N + 14\log_{2}(N) - 35 $

\section{Оценка времени}
Для замера процессорного времени исполнения функции используется функция QueryPerformanceCounter библиотеки windows.h\cite{QueryPerformanceCounter}. Проведение измерений производится в функции, приведённой в листинге 3.5.

\begin{lstlisting}[caption = Функция замера процессорного времени работы функции]
void test_time(mat_t(*f)(mat_t, mat_t, int, int, int), int n)
{
	cout << "\nРазмер матрицы: " << n << endl;
	mat_t a = random_matrix(n, n);
	mat_t b = random_matrix(n, n);
	mat_t c;
	int count = 0;
	start_counter();
	while (get_counter() < 3.0 * 1000) {
		c = f(a, b, n, n, n);
		free_mat(&c, n, n);
		count++;
	}
	double t = get_counter() / 1000;
	cout << "Выполнено " << count << " операций за " << t << " секунд" << endl;
	cout << "Время: " << t / count << endl;
	free_mat(&a, n, n);
	free_mat(&b, n, n);
}
\end{lstlisting}

