\section{Выбор языка программирования}
В качестве языка программирования был выбран C++, так как имеется опыт работы с ним, и с библиотеками, позволяющими провести исследование и тестирование программы. Также в языке имеются средства для отключения оптимизации компилятора.


\section{Листинг кода}
Реализация алгоритмов умножения матриц представлена на листингах 3.1-3.3.

\begin{lstlisting}[caption = Функция сортировки пузырьком.]
#include "bubble.h"

#pragma optimize( "", off )
void bubble_sort(arr_t& arr, int n)
{
	content_t temp;
	for (int i=1; i<n; i++)
		for (int j=0; j<n-i; j++)
			if (arr[j] > arr[j + 1])
			{
				temp = arr[j + 1];
				arr[j + 1] = arr[j];
				arr[j] = temp;
			}
}
#pragma optimize( "", on )
\end{lstlisting}

\begin{lstlisting}[caption = Функция поразрядной сортировки.]
#include "radix.h"
#pragma optimize( "", off )

int* create_digit_arr(int k)
{
	int* arr = create_arr(k);
	int n = 1;
	for (int i = 0; i < k; i++, n *= 10)
		arr[i] = n;
	return arr;
}

inline int count_k(int a)
{
	int k = 0;
	while (a > 0)
	{
		a /= 10;
		k++;
	}
	return k;
}

inline int get_digit(int a, int k, int* digits)
{
	a %= digits[k+1];
	a /= digits[k];
	return a;
}

void format_arr(arr_t arr, int n, int& decr, int& k)
{
	int min_el = arr[0];
	int max_el = min_el;
	for (int i = 1; i < n; i++)
	{
		int el = arr[i];
		if (el < min_el)
			min_el = el;
		if (el > max_el)
			max_el = el;
	}
	
	if (min_el < 0)
	{
		decr = -min_el;
		max_el -= min_el;
		for (int i = 0; i < n; i++)
			arr[i] -= min_el;
	}
	else
	{
		decr = 0;
	}
	
	k = count_k(max_el);
}

void radix_sort(arr_t& arr, int n)
{
	int k, decr;
	format_arr(arr, n, decr, k);
	int c[10]; // = { 0, 1, 2, 3, 4, 5, 6, 7, 8, 9 };
	int* digits = create_digit_arr(k+1);
	arr_t copy = create_arr(n);
	arr_t swap_arr;
	
	for (int dig = 0; dig < k; dig++)
	{
		for (int i = 0; i < 10; i++)
			c[i] = 0;
		
		for (int i = 0; i < n; i++)
			c[get_digit(arr[i], dig, digits)]++;
		
		int count = 0;
		for (int i = 0; i < 10; i++)
		{
			count += c[i];
			c[i] = count - c[i];
		}
		
		for (int i = 0; i < n; i++)
			copy[c[get_digit(arr[i], dig, digits)]++] = arr[i];
		
		swap_arr = arr;
		arr = copy;
		copy = swap_arr;
	}
	
	if (decr)
	for (int i = 0; i < n; i++)
		arr[i] -= decr;
	
	free_arr(digits);
	free_arr(copy);
}
#pragma optimize( "", on )
\end{lstlisting}

\begin{lstlisting}[caption = Функция сортировки слиянием.]
#include "merge.h"
#pragma optimize( "", off )

void merge_sort(arr_t& arr, int n)
{
	content_t temp;
	for (int i = 1; i < n; i += 2)
		if (arr[i - 1] > arr[i])
		{
			temp = arr[i - 1];
			arr[i - 1] = arr[i];
			arr[i] = temp;
		}
	
	arr_t copy = create_arr(n);
	arr_t swap_arr;
	for (int merge_size = 2; merge_size < n; merge_size *= 2)
	{
		int i = 0;
		for (; i+merge_size < n; i += merge_size * 2)
		{
			int center = i + merge_size;
			int left;
			if (center + merge_size > n)
			left = n;
			else
			left = center + merge_size;
			
			int j = i;
			int k1 = i;
			int k2 = center;
			while (k1 < center && k2 < left)
			{
				if (arr[k1] < arr[k2])
				{
					copy[j] = arr[k1];
					k1++;
				} 
				else 
				{
					copy[j] = arr[k2];
					k2++;
				}
				j++;
			}
			
			if (k1 == center)
				for (; k2 < left; k2++, j++)
					copy[j] = arr[k2];
			else 
				for (; k1 < center; k1++, j++)
					copy[j] = arr[k1];
		}
		
		if (i + merge_size >= n && i < n)
		{			
			for (int j = i; j < n; j++)
				copy[j] = arr[j];
		}
		
		swap_arr = arr;
		arr = copy;
		copy = swap_arr;
	}
	
	free_arr(copy);
}

#pragma optimize( "", on )

\end{lstlisting}


\section{Результаты тестирования}
Для тестирования написанных функций был создан отдельный файл с ранее описанными классами тестов. Тестирование функций проводилось за счёт сравнения результов функций с резельтатом функции из стандартной библиотеки algorithm.

Состав тестов приведён в листинге 3.4.

\begin{lstlisting}[caption = Модульные тесты]
#include "tests.h"
// Сравнение работы функций со стандартной
bool _cmp_funcs(arr_t arr, int n)
{
	bool flag = true;
	sort_func f_arr[] = { bubble_sort, merge_sort, radix_sort };
	arr_t std_arr = copy_arr(arr, n);
	std::sort(std_arr, std_arr + n);
	
	for (int i = 0; i < 3 && flag; i++)
	{
		arr_t my_arr = copy_arr(arr, n);
		f_arr[i](my_arr, n);
		flag = is_equal_arr(std_arr, my_arr, n);
		free_arr(my_arr);
	}
	
	free_arr(std_arr);
	return flag;
}

// Массив размером 1
void _size_one_test()
{
	std::string msg;
	msg = __FUNCTION__;		msg += " - OK";
	
	arr_t arr = create_arr(1);
	arr[0] = 42;
	
	if (!_cmp_funcs(arr, 1))
	{
		msg = __FUNCTION__;		msg += " - FAILED";
	}
	
	std::cout << msg << std::endl;
	free_arr(arr);
}
// Массив одинаковых элементов
void _same_test()
{
	std::string msg;
	msg = __FUNCTION__;		msg += " - OK";
	
	int n = 10;
	arr_t arr = create_arr(n);
	for (int i = 0; i < n; i++)
	arr[i] = 42;
	
	if (!_cmp_funcs(arr, n))
	{
		msg = __FUNCTION__;		msg += " - FAILED";
	}
	
	std::cout << msg << std::endl;
	free_arr(arr);
}
// Отсортированный массив
void _sorted_test()
{
	std::string msg;
	msg = __FUNCTION__;		msg += " - OK";
	
	int n = 100;
	arr_t arr = random_arr(n);
	std::sort(arr, arr + n);
	
	if (!_cmp_funcs(arr, n))
	{
		msg = __FUNCTION__;		msg += " - FAILED";
	}
	
	std::cout << msg << std::endl;
	free_arr(arr);
}
//  Отсортированный в обратном порядке массив
void _reverce_sorted_test()
{
	std::string msg;
	msg = __FUNCTION__;		msg += " - OK";
	
	int n = 100;
	arr_t arr = random_arr(n);
	std::sort(arr, arr + n);
	std::reverse(arr, arr + n);
	
	if (!_cmp_funcs(arr, n))
	{
		msg = __FUNCTION__;		msg += " - FAILED";
	}
	
	std::cout << msg << std::endl;
	free_arr(arr);
}

void run_tests()
{
	_size_one_test();
	_same_test();
	_sorted_test();
	_reverce_sorted_test();
}
\end{lstlisting}

Все тесты пройдены успешно.

\section{Оценка трудоёмкости}
Произведём оценку трудоёмкости алгоритов. Будем считать, что сортируется массив A$[N]$, максимальная разница двух элементов состоит из $K$ элементов.

\subsection{Алгоритм сортировки пузырьком}
	\par $ f_{bub} = 2 + (N-1)\cdot(2 + 3 + \frac{N-1 + 1}{2}\cdot[3 + 4 + 
	\left\{\begin{array}{ll}
		0, & $л.с.$\\
		9, & $х.с.$
	\end{array}\right.  ])$
	
	\par $ f_{bub} = 2 + (N-1)\cdot(5 + \frac{N}{2}\cdot[7 + 
	\left\{\begin{array}{ll}
		0, & $л.с.$\\
		9, & $х.с.$
	\end{array}\right.  ])$
	
	Лучший случай (отсортированный массив):
	\par $ f_{bub} = 2 + (N-1)\cdot(5 + \frac{N}{2}\cdot 7) = \frac{7}{2}N^{2} + \frac{3}{2}N - 3 $
	
	Худший случай (массив в обратном порядке):
	\par $ f_{bub} = 2 + (N-1)\cdot(5 + 9N) = 9N^{2} - 4N - 3$

\subsection{Алгоритм поразрядной сортировки} 
\textbf{Функция нормализации массива:}
	\par $ f_{format} = 2 + 2 + N\cdot[2 + 4 + 
		\left\{\begin{array}{ll}
			0, & $л.с.$\\
			1, & $х.с.$
		\end{array}\right.] 
	+ 2 + 
	\left\{\begin{array}{ll}
		0, & $л.с.$\\
		1 + 2 + N\cdot(2 + 2), & $х.с.$
	\end{array}\right. + 1 + (2 + 3K) $

	\par $ f_{format} = 9 + 3K + 7N +  
	\left\{\begin{array}{ll}
		0, & $л.с.$\\
		3 + 4N, & $х.с.$
	\end{array}\right. $

\textbf{Функция получения значения разряда:}
	\par $ f_{dig} = 3 + 2 = 5 $

\textbf{Функция поразрядной сортировки:}	
	\par $ f_{rad} = f_{format} + 1 + 2 +
	K\cdot(2 + 2 + 10(2 + 1) + 2 +
		N\cdot(2 + f_{dig} + 2) + 1 + 2 +
		10(2 + 6) + 2 + N\cdot(2 + f_{dig} + 5) + 3) + 1 +
	\left\{\begin{array}{ll}
		0, & $л.с.$\\
		2 + N\cdot(2 + 2), & $х.с.$
	\end{array}\right. $

	\par $ f_{rad} = 24KN + 125K + 7N + 12 + 
	\left\{\begin{array}{ll}
		0, & $л.с.$\\
		5 + 8N, & $х.с.$
	\end{array}\right.$
	
	Лучший случай (без отрицательных чисел):
	\par $ f_{rad} = 24KN + 125K + 7N + 12$
	
	Худший случай (с отрицательными числами):
	\par $ f_{rad} = 24KN + 125K + 15N + 17$
	

\subsection{Алгоритм сортировки слиянием} 
	\par $ f_{mer} = 2 + \frac{N}{2}\cdot(2 + 4 + 
	\left\{\begin{array}{ll}
		0, & $л.с.$\\
		10, & $х.с.$
	\end{array}\right.) + 1 + 2 + 
	\log_{2}(N)\cdot(2 + 2 + \dfrac{2(1-2^{\log_{2}(N)})}{(1 - 2)\cdot\log_{2}(N)}\cdot
	[4 + 2 + 1 + 
		\left\{\begin{array}{ll}
			1, & $л.с.$\\
			2, & $х.с.$
		\end{array}\right. +
		3 + 2 + 1 + \dfrac{log_{2}(N)*N}{\dfrac{2(1-2^{\log_{2}(N)})}{(1 - 2)}}\cdot
			\left\{\begin{array}{ll}
				(3 + 4 + 1)/2 + (3 + 3)/2, & $л.с.$\\
				3 + 4 + 1, & $х.с.$
			\end{array}\right.
	] + 5 + 3 + 
		\left\{\begin{array}{ll}
			0, & $л.с.$\\
			2 + \dfrac{2(1-2^{\log_{2}(N)})}{(1 - 2)\cdot\log_{2}(N)}\cdot(2 + 3), & $х.с.$
		\end{array}\right.
	)$
	\\
	
	\par $ f_{mer} = -21 + 29N + 
	\left\{\begin{array}{ll}
		0, & $л.с.$\\
		5N, & $х.с.$
	\end{array}\right. + 
	12\log_{2}(N) + 
		\left\{\begin{array}{ll}
			2N-2, & $л.с.$\\
			4N-4, & $х.с.$
		\end{array}\right. +
	(log_{2}(N)*N)\cdot
		\left\{\begin{array}{ll}
			7, & $л.с.$\\
			8, & $х.с.$
		\end{array}\right.
	 + 
	\left\{\begin{array}{ll}
		0, & $л.с.$\\
		2\log_{2}(N) + 10(N-1), & $х.с.$
	\end{array}\right.
	$ \\
	
	Лучший случай (N является степенью 2, отсортированный массив):
	\par $ f_{mer} = 7N\log_{2}(N) + 31N + 12\log_{2}(N) - 23 $
	
	Худший случай (N является степенью 2 - 1, случайный массив):
	\par $ f_{mer} = 8N\log_{2}(N) + 48N + 14\log_{2}(N) - 35 $

\section{Оценка времени}
Для замера процессорного времени исполнения функции используется функция QueryPerformanceCounter библиотеки windows.h\cite{QueryPerformanceCounter}. Проведение измерений производится в функции, приведённой в листинге 3.5.

\begin{lstlisting}[caption = Функция замера процессорного времени работы функции]
void test_time(sort_func f, int n)
{
	arr_t arr = random_arr(n);
	
	int count = 0;
	start_counter();
	while (get_counter() < 3.0 * 1000) 
	{
		fill_random_arr(arr, n, -10000, 10000);
		f(arr, n);
		count++;
	}
	double t = get_counter() / 1000;
	
	start_counter();
	for (int i = 0; i < count; i++)
	fill_random_arr(arr, n, -10000, 10000);
	t -= get_counter() / 1000;
	
	cout << "Выполнено " << count << " операций за " << t << " секунд" << endl;
	cout << "Время: " << t / count << endl;
	
	free_arr(arr);
}
\end{lstlisting}


\section*{Вывод}
Результатом технологической части стало определение требуемых программных средств, реализация алгоритмов сортировок, создание систем тестирования и замера процессорного времени работ функций.

