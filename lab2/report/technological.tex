\section{Выбор языка программирования}
В качестве языка программирования был выбран C++, так как имеется опыт работы с ним, и с библиотеками, позволяющими провести исследование и тестирование программы. Также в языке имеются средства для отключения оптимизации компилятора.


\section{Листинг кода}
Реализация алгоритмов умножения матриц представлена на листингах 3.1-3.2.

\begin{lstlisting}[caption = Функция умножения матриц классическим алгоритмом.]
#include "classic.h"
#pragma optimize( "", off )
mat_t classic_mult(mat_t a, mat_t b, int m, int n, int q)
{
	mat_t c = create_mat(m, q);
	
	for (int i = 0; i < m; i++)
	for (int j = 0; j < q; j++)
	{
		c[i][j] = 0;
		for (int k = 0; k < n; k++)
		c[i][j] += a[i][k] * b[k][j];
	}
	return c;
}
#pragma optimize( "", on )
\end{lstlisting}

\begin{lstlisting}[caption = Функция умножения матриц алгоритмом Винограда.]
#include "winograd.h"

#pragma optimize( "", off )

arr_t calc_mj(mat_t b, int n, int q)
{
	arr_t mj = create_arr(q);
	for (int j = 0; j < q; j++)
	{
		double mjj = 0; 
		for (int k = 1; k < n; k += 2)
		mjj += b[k][j] * b[k - 1][j];
		mj[j] = mjj;
	}
	return mj;
}

mat_t winograd_mult(mat_t a, mat_t b, int m, int n, int q)
{
	mat_t c = create_mat(m, q);
	arr_t mj = calc_mj(b, n, q);
	
	for (int i = 0; i < m; i++)
	{
		double mi_i = 0;
		for (int k = 1; k < n; k += 2)
		mi_i += a[i][k] * a[i][k - 1];
		
		for (int j = 0; j < q; j++)
		{
			double cij = -(mi_i + mj[j]);
			int k = 1;
			int k1 = 0;
			for (; k < n; k += 2, k1 += 2)
			cij +=	(a[i][k] + b[k1][j]) *
			(a[i][k1] + b[k][j]);
			c[i][j] = cij;
		}
	}
	
	if (n % 2)
	{
		int n_minus1 = n - 1;
		for (int i = 0; i < m; i++)
		for (int j = 0; j < q; j++)
		c[i][j] += a[i][n_minus1] * b[n_minus1][j];
	}
	
	free_arr(&mj);
	return c;
}

#pragma optimize( "", on )
\end{lstlisting}

\section{Использованые оптимизации}
Для уменьшения трудоёмкости алгоритма Винограда над его исходной версией были проделаны следующие оптимизации:
\begin{itemize}
	\item При вычислении ячеек С, промежеточный результат записывается в временную переменную, которая после получения результата переносится в матрцу С. Таким образом снижено количество операций высчитывания адреса.
	\item Цикл по слагаемым (с переменной k) изменён на аналогичный с шагом 2. Таким образом, внутри цикла не требуется производить умножение k на 2.
	\item Также введена переменная k1, равная k-1. Она, как и k, увеличивается на каждом проходе цикла на 2, и таким образом вместо 2 операций (k-1) за цикл остаётся одна.
	\item Вычисление массива multi заменено на высчитывание значения mi = multi[i] внутри общего цикла. Таким образом удалось избавиться от накладных расходов для цикла и от выделения и освобождения памяти под массив.
\end{itemize}

\section{Результаты тестирования}
Для тестирования написанных функций был создан отдельный файл с вышеописаными классами тестов. Тестирование функций проводилось за счёт сравнения результов двух функций.

Состав тестов приведён в листинге 3.3.

\begin{lstlisting}[caption = Модульные тесты]
#include "tests.h"
// Сравнение результата умножения разными способами
bool _cmp_funcs(mat_t a, mat_t b, int m, int n, int q)
{
	mat_t c1 = classic_mult(a, b, m, n, q);
	mat_t c2 = winograd_mult(a, b, m, n, q);
	bool flag = cmp_matrix(c1, c2, m, q);
	free_mat(&c1, m, q);
	free_mat(&c2, m, q);
	return flag;
}

// Матрицы с размером 1x1
void _size_one_test()
{
	mat_t a = create_mat(1, 1);
	mat_t b = create_mat(1, 1);
	
	a[0][0] = 0;
	b[0][0] = 1;
	if (!_cmp_funcs(a, b, 1, 1, 1))
	{
		std::cout << __FUNCTION__ << " - FAILED\n";
		return;
	}
	
	a[0][0] = 3;
	b[0][0] = 4;
	if (!_cmp_funcs(a, b, 1, 1, 1))
	{
		std::cout << __FUNCTION__ << " - FAILED\n";
		return;
	}
	
	free_mat(&a, 1, 1);
	free_mat(&b, 1, 1);
	
	std::cout << __FUNCTION__ << " - OK\n";
}
// Нулевые матрицы
void _void_test()
{
	mat_t a = random_matrix(3, 2);
	mat_t b = void_matrix(2, 1);
	if (!_cmp_funcs(a, b, 3, 2, 1))
	{
		std::cout << __FUNCTION__ << " - FAILED\n";
		return;
	}
	free_mat(&a, 3, 2);
	a = void_matrix(3, 2);
	if (!_cmp_funcs(a, b, 3, 2, 1))
	{
		std::cout << __FUNCTION__ << " - FAILED\n";
		return;
	}
	free_mat(&a, 3, 2);
	free_mat(&b, 2, 1);
	std::cout << __FUNCTION__ << " - OK\n";
}
// Квадратные матрицы
void _square_test()
{
	mat_t a = random_matrix(4, 4);
	mat_t b = random_matrix(4, 4);
	
	if (!_cmp_funcs(a, b, 4, 4, 4))
	{
		std::cout << __FUNCTION__ << " - FAILED\n";
		return;
	}
	
	free_mat(&a, 4, 4);
	free_mat(&b, 4, 4);
	std::cout << __FUNCTION__ << " - OK\n";
}
// Матрицы нечётного размера
void _odd_test()
{
	mat_t a = random_matrix(5, 3);
	mat_t b = random_matrix(3, 7);
	
	if (!_cmp_funcs(a, b, 5, 3, 7))
	{
		std::cout << __FUNCTION__ << " - FAILED\n";
		return;
	}
	
	free_mat(&a, 5, 3);
	free_mat(&b, 3, 7);
	std::cout << __FUNCTION__ << " - OK\n";
}

void run_tests()
{
	_size_one_test();
	_void_test();
	_square_test();
	_odd_test();
}
\end{lstlisting}

Все тесты пройдены успешно.

% //////////////
\section{Оценка трудоёмкости}
Произведём оценку трудоёмкости алгоритов. Будем считать, что умножаются матрицы $ A[M*N] $ и $ B[N*Q] $ \\

\textbf{Классический алгоритм умножения.} 
\par $ f_{cls} = 2 + M\cdot(2 + Q\cdot(3 + 2 + N\cdot(2 + 1 + 2 + 1 + 2))) $
\par $ f_{cls} = 2 + 2M + 5QM + 10MNQ $\\

\textbf{Алгоритм умножения Винограда.} 
\par $ f_{win} = (2 + Q\cdot(2 + 1 + 1 + (N/2)\cdot(2 + 1 + 2 + 1 + 3))) + 
2 + M\cdot(2 + 1 + 2 + (N/2)\cdot(2 + 1 + 2 + 1 + 2) + 2 + Q\cdot(2 + 3 + 2 + 1 + (N/2)\cdot(3 + 1 + 5 + 1 + 5))) + 1 +
\left\{\begin{array}{ll}
	0, & $л.с.$\\
	2 + 2 + M\cdot(2 + Q\cdot(2 + 3 + 2 + 1 + 2)), & $х.с.$
\end{array} \right.$\\

\par $ f_{win} = 5 + 4Q + 4.5NQ + 
7M + 4MN + 8MQ + 7.5MNQ +
\left\{\begin{array}{ll}
	0, & $л.с.$\\
	4 + 2M + 10MQ, & $х.с.$
\end{array} \right.$



\section{Оценка времени}
Для замера процессорного времени исполнения функции используется функция QueryPerformanceCounter библиотеки windows.h\cite{QueryPerformanceCounter}. Проведение измерений производится в функции, приведённой в листинге 3.4

\begin{lstlisting}[caption = Функция замера процессорного времени работы функции]
void test_time(mat_t(*f)(mat_t, mat_t, int, int, int), int n)
{
	cout << "\nРазмер матрицы: " << n << endl;
	mat_t a = random_matrix(n, n);
	mat_t b = random_matrix(n, n);
	mat_t c;
	int count = 0;
	start_counter();
	while (get_counter() < 3.0 * 1000) {
		c = f(a, b, n, n, n);
		free_mat(&c, n, n);
		count++;
	}
	double t = get_counter() / 1000;
	cout << "Выполнено " << count << " операций за " << t << " секунд" << endl;
	cout << "Время: " << t / count << endl;
	free_mat(&a, n, n);
	free_mat(&b, n, n);
}
\end{lstlisting}

