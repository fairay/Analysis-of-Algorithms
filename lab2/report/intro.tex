Трудоёмкость алгоритма - это зависимость стоимости операций от линейного размера входа\cite{perf_def}.\\
Модель вычислений трудоёмкости учитывает следующие оценки:
\begin{itemize}
	\item Оценка стоимости базовых операций. Операции =, +, - и т.д. имеют стоиость 1.
	\item Оценка циклов.
	\item Оценка условного оператора if.
\end{itemize}

Оценка характера трудоёмкости даётся по наиболее быстрорастущему слагаемому. Такая оценка играет важную роль в 
разработке и анализе алгоритмов, так как позволяет судить об оптимальности использования алгритма при тех или
иных входных данных.

В данной лабораторной оценивается трудоёмкость классического алгоритма умножения матриц и алгоритма Винограда. 
