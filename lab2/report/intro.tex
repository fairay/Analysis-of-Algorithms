Трудоёмкость алгоритма - это зависимость стоимости операций от линейного размера входа \cite{perf_def}.

Модель вычислений трудоёмкости имеет следующие оценки.
\begin{itemize}
	\item Оценка стоимости базовых операций. Базовые операции имеют стоимость 1.
	\item Оценка циклов. Включает в себя стоимость тела цикла, сравнения и инкремента.
	\item Оценка условного оператора if. Производится оценка обоих случаев.
\end{itemize}

Оценка характера трудоёмкости даётся по наиболее быстрорастущему слагаемому. Такая оценка играет важную роль в 
разработке и анализе алгоритмов, так как позволяет судить об оптимальности использования алгоритма при тех или
иных входных данных.

В данной лабораторной оценивается трудоёмкость классического алгоритма умножения матриц и алгоритма Винограда. 
