\section*{План экспериментов}
\addcontentsline{toc}{section}{Заключение}
Измерения процессорного времени проводятся при равных длинах строк s1 и s2. Содержание строк сгенерировано случайным образом. Изучается время работы при длинах: 1, 3, 10, 20, 100, 1000. Для повышения точности, каждый замер производится пять раз, за результат берётся среднее арифметическое.

% //////////////
\section*{Результат экспериментов}
\addcontentsline{toc}{section}{Результат экспериментов}
По результатам измерений процессорного времени можно составить \hyperref[table_4_1]{таблицу 4.1}

\begin{table}[h] \label{table_4_1}
\caption{Результат измерений процессорного времени (в секундах)}
\begin{tabular}{| p{2cm} | c | c | c | c | c | c |}
	\hline
	& 1				&3				&10				&20				&100		&1000 \\
	\hline\hline
	Лев., матрица	&$7*10^{-6}$	&$1.9*10^{-5}$	&$1.3*10^{-4}$	&$4.7*10^{-4}$	&$0.013$	&$1.405$ \\
	\hline
	Лев., рекурсия	&$3*10^{-6}$	&$4.7*10^{-5}$	&$6.984$		&$-$			&$-$		&$-$ \\
	\hline
	Лев., рекурсия с матрицей
	&$1*10^{-5}$	&$4.1*10^{-5}$	&$4.1*10^{-4}$	&$2.5*10^{-3}$	&$0.38$		&$-$ \\
	\hline
	Д-Л, матрица	&$8*10^{-6}$	&$2.8*10^{-5}$	&$1.7*10^{-4}$	&$6.1*10^{-4}$	&$0.016$	&$2.031$ \\
	\hline
\end{tabular}
\end{table}

В алгоритме нахождения расстояния Левенштейна с помощью рекурсии замеры на длине строк более 10 не проводились, так как время выполнения было слишком велико (более 10 минут). В алгоритме рекурсии с заполнением матрицы не удалось провести измерения при длине 1000, так как была превышена максимальная глубина рекурсии.

% //////////////
\section*{Сравнительный анализ}
\addcontentsline{toc}{section}{Сравнительный анализ}
По результатам эксперимента можно заключить следующее.
\begin{itemize}
	\item Наиболее быстродейственным алгоритмом поиска расстояния Левенштейна является алгоритм, использующий матрицу.
	\item Рекурсивный алгоритм с использованием матрицы показывает значительно более низкую скорость роста времени по сравнению с рекурсивным алгоритмом.
	\item Алгоритмы поиска расстояний Левенштейна и Дамерау-Левенштейна с помощью матрицы показывают схожую скорость роста времени, однако первый алгоритм несколько быстрее.
\end{itemize}


	