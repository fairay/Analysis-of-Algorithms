\section{План экспериментов}
\addcontentsline{toc}{section}{Заключение}
Измерения процессорного времени проводятся на квадратных матрица. Содержание матриц сгенерировано случайным образом. Ввиду разного поведения алгоритма Винограда для чётных и нечётных размерностей, время работы изучается двумя сериями экспериментов с размерностями матриц:
\begin{enumerate}
	\item 50, 100, 200, 400, 800;
	\item 51, 101, 201, 401, 801.
\end{enumerate}

Для повышения точности, каждый замер производится пять раз, за результат берётся среднее арифметическое.


\section{Результат экспериментов}
\addcontentsline{toc}{section}{Результат экспериментов}
По результатам измерений процессорного времени можно составить \hyperref[table_4_1]{таблицу 4.1} и \hyperref[table_4_2]{таблицу 4.2}

\begin{table}[h] \label{table_4_1}
\caption{Чётная размерность матриц. Результат измерений процессорного времени (в секундах)}
\begin{tabular}{| p{3.5cm} | c | c | c | c | c | c |}
	\hline
					& 50				&100				&200			&400		&800	\\
	\hline\hline
	Классический	&$5.1\cdot10^{-4}$	&$4.2\cdot10^{-3}$	&$0.037$		&$0.32$		&$3.54$	\\
	\hline
	Виноград		&$3.2\cdot10^{-4}$	&$2.7\cdot10^{-3}$	&$0.023$		&$0.20$		&$2.31$	\\
	\hline
\end{tabular}
\end{table}


\begin{table}[h] \label{table_4_2}
	\caption{Нечётная размерность матриц. Результат измерений процессорного времени (в секундах)}
	\begin{tabular}{| p{3.5cm} | c | c | c | c | c | c |}
		\hline
						& 51				&101				&201			&401		&801	\\
		\hline\hline
		Классический	&$5.0\cdot10^{-4}$	&$4.1\cdot10^{-3}$	&$0.034$		&$0.35$		&$3.48$	\\
		\hline
		Виноград		&$3.3\cdot10^{-4}$	&$2.5\cdot10^{-3}$	&$0.023$		&$0.21$		&$2.27$	\\
		\hline
	\end{tabular}
\end{table}

Эксперименты проводились на компьютере с характеристиками:
\begin{itemize}
	\item ОС - Windows 10, 64 бит;
	\item Процессор -  Intel Core i7 8550U (1800 МГц);
	\item Объем ОЗУ: 8 ГБ.
\end{itemize}

% //////////////
\section{Сравнительный анализ}
\addcontentsline{toc}{section}{Сравнительный анализ}
По результатам экспериментов можно заключить следующее.
\begin{itemize}
	\item Алгоритм Винограда затрачивает меньше времени, чем классический алгоритм умножения на всех исследованных размерах матриц.
	\item Существенных различий в процессорном времени при умножении матриц чётных и нечётных размеров у алгоритмп Винограда не выявлено.
	\item При увеличении размера матриц в 2 раза, наблюдается рост затраченного процессорного времени для обоих алгоритмов примерно в 8-10 раз, что соответсвует расчётам их трудоёмкости.
\end{itemize}


	