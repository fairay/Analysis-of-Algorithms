Целью лабораторной работы является оценка трудоёмкости алгоритма умножения матриц и получение
практического навыка оптимизации алгоритмов.

Выделены следующие задачи лабораторной работы:

\begin{itemize}
\item математическое описание операции умножения матриц;
\item описание и реализация алгоритмов умножения матриц;
\item описание применённых к алгоритму Винограда способов оптимизации;
\item проведение замеров процессорного времени работы алгоритмов при различных размерах матриц 
(серия экспериментов для чётного размера и для нечётного);
\item оценка трудоёмкости алгоритов;
\item проведение сравнительного анализа алгоритмов на основании экспериментов.
\end{itemize}

Умножение матриц - операция над матрицами А$[M*N] $ и B$[N*Q]$. Результатом операции является матрица C размерами $ M*Q $, в которой элемент $c_{i,j}$ задаётся формулой
\begin{equation} 
	c_{i,j} = \sum_{k=1}^{N}(a_{i,k} \cdot b_{k,j})
\end{equation}

Для оптимизиции трудоёмкости умножения матриц таким методом можно воспользоваться следующими соотношениями.

Пусть $ u, v $ - элементы матриц А, B соотв., участвующие в вычислении значения элемента матрицы C. Тогда данный элемент вычисляется как 
\begin{equation} 
	u_{1}v_{1} + u_{2}v_{2} + u_{3}v_{3} + u_{4}v_{4}
\end{equation}

Такое выражение можно представить как 
\begin{equation} 
	(u_{1} + v_{2})(v_{1} + u_{2}) + (u_{3} + v_{4})(v_{3} + u_{4}) - u_{1}u_{2} - u_{3}u_{4} - v_{1}v_{2} - v_{3}v_{3}
\end{equation}

В этом выражении вычитаемые можно вычислить однократно и применить их для всех столбцов и строк, где они используются. Данные действия испольняются в алгоритме Винограда. Таким образом удаётся снизить трудоёмкость алгоритма за счёт снижения количества операций. В случае, если матрица нечётный размер N, требуется производить дополнительные вычисления для крайних строк и столбцов. Поэтому алгоритм наиболее эффективен в случае матриц, у которых N является чётным.

