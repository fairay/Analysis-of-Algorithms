Целью лабораторной работы является оценка трудоёмкости алгоритма умножения матриц и получение
практического навыка оптимизации алгоритмов.

Выделены следующие задачи лабораторной работы:

\begin{itemize}
\item математическое описание операции умножения матриц;
\item описание и реализация алгоритмов умножения матриц;
\item описание применённых к алгоритму Винограда способов оптимизации;
\item проведение замеров процессорного времени работы алгоритмов при различных размерах матриц 
(серия экспериментов для чётного размера и для нечётного);
\item оценка трудоёмкости алгоритом;
\item проведение сравнительного анализа алгоритмов на основании экспериментов.
\end{itemize}

Умножение матриц - операция над матрицами $ А[MxN] $ и $ B[NxQ] $. Результатом операции является матрица C размерами $ M*Q $, в которой элемент $c_{i,j}$ задаётся формулой
\begin{equation} 
	c_{i,j} = \sum_{k=1}^{N}(a_{i,k} \cdot b_{k,j})
\end{equation}

