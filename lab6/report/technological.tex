\section{Выбор языка программирования}
В качестве языка программирования был выбран C++\cite{C++_Doc}, так как имеется опыт работы с ним, и с библиотеками, позволяющими провести исследование и тестирование программы. Разработка проводилась в среде Visual Studio 2019\cite{VisualStudio}.

\section{Листинги кода}
Реализация алгоритмов поиска представлена на листингах 3.1-3.3.

\begin{lstlisting}[caption = {Поиск полным перебором}]
path_t explore_brunch(const len_matrix& m, const path_t& available_nodes, size_t cur_node, len_t& len)
{
	path_t res_path;
	len = -1;
	
	if (!available_nodes.size())
	{
		if (m[cur_node][0] < 0)
			return res_path;
		
		len = m[cur_node][0];
		res_path.push_back(0);
	}
	else
	{
		for (size_t i = 0; i < available_nodes.size(); i++)
		{
			size_t next_node = available_nodes[i];
			if (m[cur_node][next_node] < 0)
				continue;
			
			len_t temp_len = -1;
			path_t temp_nodes = available_nodes;
			temp_nodes.erase(temp_nodes.begin() + i);
			
			path_t temp_path = explore_brunch(m, temp_nodes, next_node, temp_len);
			if (temp_len < 0)
				continue;
			temp_len += m[cur_node][next_node];
			
			if (len < 0 || len > temp_len)
			{
				res_path = temp_path;
				len = temp_len;
			}
		}
	}
	
	res_path.push_back(cur_node);
	return res_path;
}

path_t brute_force(const len_matrix& m, len_t& len)
{
	path_t full_p(all_nodes(m));
	full_p.erase(full_p.begin()); // delete 0 node
	len = -1;
	path_t ans(explore_brunch(m, full_p, 0, len));
	if (ans.size() == m.size() + 1)
		return ans;
	else
	{
		len = 0;
		return path_t();
	}
}
\end{lstlisting}

\begin{lstlisting}[caption = {Поиск муравьиным алгоритмом (главная функция)}]
path_t ant_search(const len_matrix& m, ant_config& cnf)
{
	double init_tau = 1;
	double min_tau = init_tau/10;
	vector<vector<double>> tau = create_matrix(m.size(), init_tau);
	
	path_t min_path;
	len_t min_len = -1;
	size_t elite_n = 4;
	
	for (size_t t = 0; t < cnf.max_t; t++)
	{	
		ant_arr colony = init_colony(m);
		vector<vector<double>> d_tau = create_matrix(m.size(), 0);
		
		for (size_t i = 0; i < colony.size(); i++)
		{
			ant_t& ant = colony[i];
			size_t init_pos = ant.pos;
			
			while (ant.avl_nodes.size())
				if (!next_step(ant, m, tau, cnf)) 
					break;
			
			if (ant.avl_nodes.size())
			{ 
				ant.temp_len = -1;
			}
			else
			{
				ant.avl_nodes.push_back(init_pos);
				if (!next_step(ant, m, tau, cnf)) 
					ant.temp_len = -1;
			}
		}
		
		for (ant_t ant : colony)
		{
			if (ant.temp_len < 0)
				continue;

			if (min_len > ant.temp_len || min_len < 0)
			{
				min_len = ant.temp_len;
				min_path = ant.path;
			}
			
			double inc = ((double)cnf.q) / ant.temp_len;
			for (size_t i = 1; i < ant.path.size(); i++)
			{
				d_tau[ant.path[i]][ant.path[i-1]] += inc;
				d_tau[ant.path[i-1]][ant.path[i]] += inc;
			}
		}
		
		double inc = ((double)cnf.q) / min_len * elite_n;
		for (size_t i = 1; i < min_path.size(); i++)
		{
			d_tau[min_path[i]][min_path[i - 1]] += inc;
			d_tau[min_path[i - 1]][min_path[i]] += inc;
		}
		
		for (size_t i = 0; i < tau.size(); i++)
			for (size_t j = 0; j < tau.size(); j++)
				tau[i][j] = max(tau[i][j]*(1 - cnf.ro) + d_tau[i][j], min_tau);
	}
	
	return min_path;
}
\end{lstlisting}

\begin{lstlisting}[caption = {Поиск муравьиным алгоритмом (дополнительные функции)}]
ant_arr init_colony(const len_matrix& m)
{
	size_t len = m.size();
	path_t pos = all_nodes(m);
	random_shuffle(pos.begin(), pos.end());
	
	ant_arr arr(len);
	for (size_t i = 0; i < len; i++)
	{
		arr[i].pos = pos[i];
		arr[i].temp_len = 0;
		arr[i].path.push_back(pos[i]);
		
		arr[i].avl_nodes = all_nodes(m);
		arr[i].avl_nodes.erase(arr[i].avl_nodes.begin() + pos[i]);
	}
	
	return arr;
}

int get_next_node(const ant_t& ant, const len_matrix& m, const vector<vector<double>>& tau, const ant_config& cnf)
{
	size_t cur = ant.pos;
	vector<double> node_p(ant.avl_nodes.size(), 0);
	double sum_p = 0;
	
	for (size_t j = 0; j < ant.avl_nodes.size(); j++)
	{
		size_t next = ant.avl_nodes[j];
		if (m[cur][next] < 0)
			continue;
		
		double val = pow(tau[cur][next], cnf.a) / pow(m[cur][next], cnf.b);
		sum_p += val;
		node_p[j] = sum_p;
	}
	if (sum_p < 1e-9) 
		return -1; 
	
	double rand_f = ((double)rand() / RAND_MAX) * sum_p * (1 - 1e-8);
	
	for (size_t next = 0; next < node_p.size(); next++)
		if (node_p[next] > rand_f)
			return ant.avl_nodes[next];
	return ant.avl_nodes[node_p.size() - 1];
}
int next_step(ant_t& ant, const len_matrix& m, const vector<vector<double>>& tau, const ant_config& cnf)
{
	size_t cur = ant.pos;
	int next = get_next_node(ant, m, tau, cnf);
	if (next == -1)	return 0;
	
	ant.pos = next;
	ant.temp_len += m[cur][next];
	ant.path.push_back(next);
	ant.avl_nodes.erase(find(ant.avl_nodes.begin(), ant.avl_nodes.end(), next));	
	
	return 1;
}
\end{lstlisting}

\section{Автоматическая параметризация муравьиного алгоритма}
Для исследования работы муравьиного алгоритма на разных наборах функций была написана функция автоматической параметризации, приведённая в листинге 3.4.

\begin{lstlisting}[caption = {Функции автоматической параметризации муравьиного алгоритма}]
void best_config(const len_matrix& m, len_t q, len_t perfect_len)
{
	len_t min_len = q * 1000;
	ant_config best_cnf;
	for (double a = 0; a <= 1; a+=0.1)
	{
		double b = 1 - a;
		for (double ro = 0; ro <= 1; ro += 0.1)
		{
			ant_config cnf = create_config(a, ro, 30, q);
			len_t local_min = q * 1000;
			for (int i=0; i<3; i++)
			{
				path_t p = ant_search(m, cnf);
				len_t len = path_len(m, p);
				local_min = min(len, local_min);
			}
			printf("%.1lf %.1lf %.1lf %zd: %.2lf %.2lf\n", a, b, ro, cnf.max_t,	local_min, local_min - perfect_len);
			if (min_len > local_min)
			{
				min_len = local_min;
				best_cnf = cnf;
			}
		}
		printf("\n");
	}
	
	printf("%.1lf %.1lf %.1lf : %.2lf\n", best_cnf.a, best_cnf.b, best_cnf.ro, min_len);
}
\end{lstlisting}

\section{Результаты тестирования}
Для тестирования написанных функций был создан отдельный файл с ранее описанными классами тестов. Тестирование функций проводилось за счёт сравнения результов функций c ожидаемым результатом. Отдельно стоит отметить, что тестирование муравьиного алгоритма в общем случае затруднено непредсказуемостью ответа из-за случайной составляющей.

Состав тестов приведён в листинге 3.5.

\begin{lstlisting}[caption = {Модульные тесты}]
#include "tests.h"
using namespace std;
bool _no_way()
{
	cout << __FUNCTION__;
	len_t len = 0;
	path_t path;
	len_matrix m = random_matrix(7, 1, 9, 0.99);
	
	len = 0;
	path = brute_force(m, len);
	if (len || path.size()) return false;
	
	ant_config cnf = create_config(0.5, 0.5, 20, calculate_q(m));
	path = ant_search(m, cnf);
	if (path.size())    return false;
	return true;
}

bool _same_way()
{
	cout << __FUNCTION__;
	len_t len = 0;
	path_t path;
	len_matrix m = random_matrix(7, 1, 1);
	
	len = 0;
	path = brute_force(m, len);
	if (len != m.size()|| path.size() != m.size() + 1) return false;
	
	ant_config cnf = create_config(0.5, 0.5, 20, calculate_q(m));
	path = ant_search(m, cnf);
	if (path_len(m, path) != m.size() || path.size() != m.size() + 1)    return false;
	return true;
}

bool _size_two()
{
	cout << __FUNCTION__;
	len_t len = 0;
	path_t path;
	len_matrix m = random_matrix(2, 1, 9);
	
	len_t ans = m[1][0] + m[0][1];
	len = 0;
	path = brute_force(m, len);
	if (len != ans || path.size() != 3) return false;
	
	ant_config cnf = create_config(0.5, 0.5, 20, calculate_q(m));
	path = ant_search(m, cnf);
	if (path.size() != 3 || path_len(m, path) != ans)    return false;
	return true;
}

bool _rnd_matrix()
{
	cout << __FUNCTION__;
	len_t len = 0;
	path_t path;
	len_matrix m = random_matrix(10, 1, 9);
	
	len = 0;
	path = brute_force(m, len);
	if (path.size() != 11) return false;
	
	return true;
}


using test_f = bool(*)(void);
void run_tests()
{
	cout << "Running tests:" << endl;
	
	test_f f_arr[] = { _no_way, _same_way, _size_two, _rnd_matrix };
	
	for (size_t i = 0; i < 4; i++)
	{
		if (f_arr[i]())
			cout << " - PASSED\n";
		else
			cout << " - FAILED\n";
	}
	
	cout << endl;
}
\end{lstlisting}

\section{Оценка времени}
Для замера процессорного времени исполнения функции используется функция QueryPerformanceCounter библиотеки windows.h\cite{QueryPerformanceCounter}. Код функций замера времени приведёны в листинге 3.6.

\begin{lstlisting}[caption = {Функции замера процессорного времени работы функции}]
double PCFreq = 0.0;
__int64 CounterStart = 0;

void start_counter()
{
	LARGE_INTEGER li;
	QueryPerformanceFrequency(&li);
	
	PCFreq = double(li.QuadPart) / 1000.0;
	
	QueryPerformanceCounter(&li);
	CounterStart = li.QuadPart;
}

double get_counter()
{
	LARGE_INTEGER li;
	QueryPerformanceCounter(&li);
	return double(li.QuadPart - CounterStart) / PCFreq;
}
\end{lstlisting}

\section*{Вывод}
Результатом технологической части стал выбор используемых технических средств реализации и реализация алгоритмов, системы тестов и замера времени работы на языке С++.
