В данной лабораторной реализуются и оцениваются алгоритмы решения задачи коммивояжёра.

Задача коммивояжёра является одним из самых известных примеров NP-полной задачи. Она заключается, в поиске наиболее выгодного маршрута, проходящего однократно через все вершины графа, кроме начальной вершины, которая должна оказаться и конечной. 

Интерес к данной задачи обусловлен тем, что на данный момент не существует алгоритма, способного находить её решение за полиномиальное время в зависимости от количества вершин. При этом, известны алгоритмы, которые способны найти маршрут, который по длине будет достаточно приближён к по наилучшему решению. Этот факт позволяет принимать подобные решения в практике, когда "почти идеальное" решение более чем удволетворяет требованиям решаемой проблемы. Примером подобной задачи является поиск маршрута пайки печатной платы, при котором манипулятор-пайщик проделает наименьший путь между контактами. В данном случае возможно использование достаточно короткого, но не лучшего маршрута.

В данной лабораторной работе в качестве алгоритмов поиска решения будут рассмотрены:
\begin{itemize}
	\item поиск полным перебором;
	\item поиск муравьиным алгоритмом.
\end{itemize}

В первом случае будет измерения длины всех возможных маршрутов. Это является достаточно затратным решением, но гарантированно будет получено наилучшее решение.

Второй алгоритм является воплощением механизма, созданого самой природой -- поведением колонией муравьёв. Суть поведения каждого муравья заключается в использовании опыта ранее ходивших муравьёв в принятии решения о выборе следующей вершины. Опыт задаётся при помощи откладывания на рёбрах графа феромона, который тем больше, чем оптимальнее маршрут проходящий через данное ребро. Таким образом, спустя множество поколений муравьёв, пользующихся знаниями своих предков, можно выявить наиболее оптимальный вариант прохождения.
