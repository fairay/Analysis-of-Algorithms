\section{Описание эксперимента}
Измерения процессорного времени проводятся на словаре размером $1500$. Содержание сгенерировано случайным образом. Вычисляются и демонстрируются минимальное, максимальное и среднее время поиска ключа, а также время поиска несуществующиего ключа.

Для повышения точности, каждый замер производится пять раз, за результат берётся среднее арифметическое.

\section{Результат эксперимента}
По результатам измерений процессорного времени можно составить таблицу
\hyperref[table_4_1]{4.1}


\begin{table}[h] \label{table_4_1}
	\caption{Результат измерений процессорного времени (в микросекундах)}
	\begin{tabular}{| p{5.0cm} | c | c | c | c | c | c |}
		\hline
		Алгоритм \ Время	&Минимальное	&Максимальное	&Среднее	&$\nexists$	\\
		\hline\hline
		Полный перебор		&$0.021$		&$1.11$			&$0.38$		&$1.42$\\
		\hline
		Половинное деление	&$0.023$		&$0.089$		&$0.048$	&$0.074$\\
		\hline
		По сегментам		&$0.023$		&$0.12$			&$0.067$	&$0.14$\\
		\hline
	\end{tabular}
\end{table}

\section{Характеристики ПК}
Эксперименты проводились на компьютере с характеристиками:
\begin{itemize}
	\item ОС - Windows 10, 64 бит;
	\item Процессор -  Intel Core i7 8550U (1800 МГц, 4 ядра, 8 логических процессоров);
	\item Объем ОЗУ: 8 ГБ.
\end{itemize}

% //////////////
\section*{Вывод}
По результатам экспериментов можно заключить следующее.
\begin{itemize}
	\item Наименьшее минимальное время показывает функция полного перебора. Это объясняется тем, что в других алгоритмах трудоёмкость лучшего случая сделана больше из-за стремления уменьшить трудоёмкость худшего и среднего случая.
	\item Также полный перебор показывает и наибольшее максимальное время поиска на порядок превышающее значения у других алгоритмов.
	\item Наиболее быстродейственным как в худшем, так и в среднем случае оказался алгоритм половинного деления.
	\item Алгоритм поиска по сегментам продемонстрировал время в среднем и худшем случае медленне лишь на $50\%$ по сравнению с методом половинного деления, что говорит о том, что этот способ оптимизации также является достаточно эффективным средством ускорения процедуры поиска.
	\item Во всех случаях время поиска несуществующего ключа примерно равно максимальному времени поиска.
\end{itemize}


	