\section{Цель и задачи работы}
Целью лабораторной работы является проведение сравнительного анализ метода полного перебора и эвристического метода на базе муравьиного алгоритма.

Выделены следующие задачи лабораторной работы:

\begin{itemize}
\item описание задачи коммивояжёра;
\item описание и реализация метода полного перебора и метода на базе муравьиного алгоритма для решения задачи коммивояжёра;
\item проведение параметризации муравьиного метода (определение параметров, для которых метод даёт наилучшие результаты на выбранных классах задач).
\end{itemize}

\section{Описание задачи коммивояжёра}
Задача заключается в поиске гамильтонова цикла (т.е. замкнутый путь, проходящий через каждую вершину ровно один раз) на неориентированном графе $G$, с количеством вершин $N$\cite{Corman}. Вес рёбер можно задать с помощью квадратной матрицы $D$ размером $N$, где $D_{ij}$ равняется стоимости перехода из вершины $i$ в $j$. Сами маршруты можно представить как массив $M$ длиной $N+1$, где $M_{i}$ - вершина, посещённая в $i$-ю очередь.

\section{Поиск полным перебором}
Данный алгоритм составляет все возможные маршруты, начинающиеся из нулевой вершины, и измеряет длину каждого из вариантов. Маршрут с минимальной длиной гарантированно будет являться решением поставленной задачи. 

Каждый маршрут начинается и заканчивается в нулевой вершине, потому что каждый путь обязательно будет содержать эту вершину, а так как это цикл, то любая последовательность посещения вершин может быть преобразована в маршрут из нулевой вершины, обладающий той же длиной. Поэтому, не имеет смысла рассмотрение иных начальных вершин.

\section{Поиск муравьиным алгоритмом}
Алгоритм симулирует поведение $N$ муравьёв, которые вместе называются колонией. Колония существует $max_t$ дней. В начале $t$-го дня по всем вершинам выставляется по муравью. Каждый муравей совершает попытку построить гамельтонов цикл. В случае удачного построения цикла на пройденых рёбрах им выставляется определённое количество феромона. После этого наступает $t$-я ночь, в которой часть феромона улетучивается, после чего начинается следующий день. Количество феромона на ребре $i-j$ в момент времени t обозначается как $\tau_{ij}(t)$

После симуляции всех дней алгоритм выдаёт в качестве решения наикратчайший маршрут среди всех пройденых. Стоит оговорить то, далеко не всегда этот маршрут будет являться правильным решением поставленной задачи, так как вероятнее всего, алгоритм проверит все возможные пути в графе.

Рассмотрим принцип формирования маршрута. Оказываясь в очередной вершине $i$ (кроме заключительной), муравей совершает выбор одной из доступных для перехода вершин, которые ещё не были посещены. Выбор основывается на величине, определяемой формулой \ref{posible}
\begin{equation}\label{posible}
	P_{ij}(t) = [\tau_{ij}(t)]^{\alpha} / [D_{ij}]^{\beta}
\end{equation} где
	$\alpha, \beta$ - коэффициенты стадности и жадности.

Полученные значения можно пронормировать, поделив их на сумму всех величин, в таком случае получится вероятность перехода в $j$-ю вершину. Основываясь на этом муравей делает случайный выбор следующей вершины с заданными вероятностями.

Каждый муравей прошедший полный маршрут увеличивает значение феромона в посещённых рёбрах на величину, указанной в формуле \ref{phero_inc}
\begin{equation}\label{phero_inc}
	\Delta\tau_{ijk}(t) = Q / L_{k}(t)
\end{equation} где
	$k$ - номер муравья,
	$Q$ - параметр, по порядку приближенный к ожидаемой минимальной длине пути,
	$L_{k}(t)$ - путь пройденый k-м муравьём в день t,
	$\tau_{ijk}(t)$ - приращение феромона от k-го муравья в день t
	
Для акцентирования феромонов на лучшем пути также используется $EL$ элитных муравьёв, которые каждый проходят по наикратчайшему маршруту на данный момент и, как и обычные муравьи, оставляют феромоны по формуле\ref{phero_inc}

После учёта всех приращений происходит испарение феромона, т.е. значение феромона в следующий день вычисляется как \ref{phero_dec}
\begin{equation}\label{phero_dec}
	\tau_{ij}(t+1) = (1-\rho)\tau_{ij}(t) + \sum\limits_{k=1}^{N} \Delta\tau_{ijk}(t)
\end{equation} где
	$\rho$ - коэффициент испарения феромона ($\rho \in [0,1]$).


\section*{Вывод}
Результатом аналитического раздела стало определение цели и задач работы, описана задача коммивояжёра и алгоритмы поиска.