В ходе лабораторной работы достигнута поставленная цель: проведён сравнительный анализ метода полного перебора и эвристического метода на базе муравьиного алгоритма.

Была изучена и описана задача коммивояжёра. Также были реализован метод полного перебора и метод на базе муравьиного алгоритма для решения задачи коммивояжёра. Проведены замеры процессорного времени и оценена трудоёмкость муравьиного алгоритма. Также проведена параметризация муравьиного метода и на основании полученных результатов проведён сравнительный анализ.

Из проведённых экспериментов можно заключить следующее. Алгоритм полного перебора всегда выдаёт правильное решение, но область его применимости ограничена графами, количество вершин которого не превышает 15. Далее, время вычисления ответа на любом устройстве будет неудволетвроительно долгим практически для любой задачи.

Алгоритм муравьиного поиска не является абсолютно точным, однако при правильно подобраных параметрах он способен находить точное решение, или маршрут, длина которого будет отличаться от эталонной на незначительную величину. При этом сложность алгоритма составляет примерно $O(n^{3})$, поэтому он способен работать с графами, размер которых существенно превышает ограничение озвученное для предыдущего алгоритма.
