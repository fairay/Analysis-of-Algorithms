В данной лабораторной реализуется и оценивается конвейерный алгоритм шифрования.

Конвейеризация алгоритма --- выполнение нескольких последовательных стадий одной задачи в разных потоках. Так же, как и в случае параллелизма, подобное разделение стадий задач между потоками нацелено на получение выигрыша за счёт их параллельного выполнения. Зачастую подобное решение применяется, когда процессоры комьютера оптимизированы под выполнения различного рода операций, что требует распределения стадий задачи.

Алгоритмы шифрования предназначены для защиты данных от прочтения строронними лицами. Существует множество разоичных алгоритмов шифрования, и чтобы обеспечить лучшую защищённость данных иногда прибегают к последовательному шифрованию данных различными алгоритмами. Так как в данной задаче явно прослеживается несколько стадий с сопоставимым временем выполнения, она подходит для реализации программного конвейера.