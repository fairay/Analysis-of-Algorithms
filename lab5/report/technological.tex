\section{Выбор языка программирования}
В качестве языка программирования был выбран C++\cite{C++_Doc}, так как имеется опыт работы с ним, и с библиотеками, позволяющими провести исследование и тестирование программы. Также в языке имеются средства для использования многопоточности\cite{Thread}. Разработка проводилась в среде Visual Studio 2019\cite{VisualStudio}.

\section{Листинги кода}
Реализация алгоритмов шифрования строк представлена на листингах 3.1-3.3. Функции конвейера и ленты приведены на листингах 3.4-3.5

\begin{lstlisting}[caption = {Функция шифра Фейстеля}]
void FeistelCipher(string& s)
{
	const int max_step = 4;
	int l, r, temp;
	
	for (int i = 1; i < s.length(); i += 2)
	{
		l = s[i - 1];
		r = s[i];
		
		for (int st = 0; st < max_step; st++)
		{
			temp = l;
			l = r ^ ((l + st * 15) % 256);
			r = temp;
		}
		s[i - 1] = l;
		s[i] = r;
	}
}
\end{lstlisting}

\begin{lstlisting}[caption = {Функция шифра Тритемиуса}]
void TrithemiusCipher(string& s)
{
	for (int i = 0; i < s.length(); i++)
	{
		s[i] = (s[i] + (int(sqrt(i)) - 3 * i + 3)) % 256;
	}
}
\end{lstlisting}

\begin{lstlisting}[caption = {Функция шифра побитовыми сдвигами}]
using buf_t = long unsigned int;
void BitShiftCipher(string& s)
{
	buf_t buf;
	int buf_byte = sizeof(buf_t);
	int buf_bit = buf_byte * 8;
	int shift = 3;
	
	for (int i = 0; i < s.size() - buf_byte + 1; i++)
	{
		memcpy(&buf, &s[i], buf_byte);
		buf = (buf >> shift) | (buf << (buf_bit - shift));
		memcpy(&s[i], &buf, buf_byte);
	}
}
\end{lstlisting}

\begin{lstlisting}[caption = {Функция основного потока (конвейера)}]
void process_data(conv_queue& task_queue)
{
	int queue_len = task_queue.size();
	conv_shared task;
	
	double t_wait, t_proc;
	double min_wait, max_wait, sum_wait;
	double min_proc, max_proc, sum_proc;
	min_wait = 1e10;    min_proc = 1e10;
	max_wait = -1;      max_proc = -1;
	sum_wait = 0;       sum_proc = 0;
	for (int i = 0; i < queue_len; i++)
	{
		task = task_queue.front();
		task_queue.pop();
		t_wait = (task->t_start[1] - task->t_end[0]) + 
		(task->t_start[2] - task->t_end[1]);
		min_wait = min(min_wait, t_wait);
		max_wait = max(max_wait, t_wait);
		sum_wait += t_wait;
		
		t_proc = task->t_end[2] - task->t_start[0];
		min_proc = min(min_proc, t_proc);
		max_proc = max(max_proc, t_proc);
		sum_proc += t_proc;
	}
	
	cout << endl << endl;
	printf("WAITING TIME:\t Min:%6.2lf\t Max:%6.2lf\t Avg:%6.2lf\n",
	min_wait, max_wait, sum_wait / queue_len);
	printf("PROCESS TIME:\t Min:%6.2lf\t Max:%6.2lf\t Avg:%6.2lf\n",
	min_proc, max_proc, sum_proc / queue_len);
}

void main_thread(int queue_len)
{
	fCipher f_arr[] = { FeistelCipher, TrithemiusCipher, BitShiftCipher };
	vector<thread> thread_arr;
	conv_queue q_arr[4];
	q_arr[0] = generate_queue(queue_len, 1000*1000);
	
	start_counter();
	for (int i = 0; i < 3; i++)
	thread_arr.push_back(thread(conv_thread,
	ref(q_arr[i]), ref(q_arr[i+1]), f_arr[i], queue_len, i));
	
	for (int i = 0; i < 3; i++)
	thread_arr[i].join();
	
	process_data(q_arr[3]);
}
\end{lstlisting}

\begin{lstlisting}[caption = {Функция рабочего потока (ленты)}]
void conv_thread(conv_queue& pre_queue, conv_queue& post_queue, fCipher f, int queue_len, int conv_n)
{
	conv_shared task;
	for (int i = 0; i < queue_len; i++)
	{
		while (pre_queue.empty()) {}
		task = pre_queue.front();
		task->t_start[conv_n] = get_counter();
		if (conv_n == 0)
		{
			pre_queue.pop();
		}
		else 
		{
			mtx.lock();
			pre_queue.pop();
			mtx.unlock();
		}
		
		f(task->data);
		
		task->t_end[conv_n] = get_counter();
		if (conv_n == 2)
		{
			post_queue.push(task);
		}
		else
		{
			mtx.lock();
			post_queue.push(task);
			mtx.unlock();
		}
	}
}
\end{lstlisting}


\section{Результаты тестирования}
Ввиду крайне ограниченой зависимости от входных данных, тестирование проводилось вручную путём введения вышевописанных случаев через консоль. Все тесты пройдены успешно.

\section{Оценка времени}
Для замера процессорного времени исполнения функции используется функция QueryPerformanceCounter библиотеки windows.h\cite{QueryPerformanceCounter}. Код функций замера времени приведёны в листинге 3.6.

\begin{lstlisting}[caption = {Функции замера процессорного времени работы функции}]

double PCFreq = 0.0;
__int64 CounterStart = 0;

void start_counter()
{
	LARGE_INTEGER li;
	QueryPerformanceFrequency(&li);
	
	PCFreq = double(li.QuadPart) / 1000.0;
	
	QueryPerformanceCounter(&li);
	CounterStart = li.QuadPart;
}

double get_counter()
{
	LARGE_INTEGER li;
	QueryPerformanceCounter(&li);
	return double(li.QuadPart - CounterStart) / PCFreq;
}
\end{lstlisting}

\section*{Вывод}
Результатом технологической части стал выбор используемых технических средств реализации и реализация алгоритмов, системы тестов и замера времени работы на языке С++.
