\section{Цель и задачи работы}
Целью лабораторной работы является разработка и исследование конвейерного алгоритма шифования строк.

Выделены следующие задачи лабораторной работы:

\begin{itemize}
\item описание понятия конвейеризации и операции шифрования;
\item описание и реализация конвейерного алгоритма шифрования строк;
\item проведение замеров времени работы и простоя конвейера;
\end{itemize}

\section{Описание операции шифрования}
Шифрование - операция обратимного изменения вида сообщения в целях сокрытия от неавторизованных лиц. Результатом операции является набор данных отличный от исходного, но имеющий метод преобразования к нему (дешифрование). В данной лабораторной работе исходные и зашифрованные данные представляются в виде символьной строки.

\section{Используемые алгоритмы}
В лабораторной работе был использован шифр, использующий последовательное шифрование тремя алгоритмами\cite{kripto}. Алгоритмы были выбраны ввиду относительно схожего времени работы на одинаковых строках. Конвейеризация может быть применена, так как существует явное разделение шифрования на последовательные этапы. Поэтому конвейер будет обладать тремя лентами, каждая из которых будет выполнять один из алгоритмов шифрования.

\subsection{Шифр Фейстеля}
	Основной операцией метода служит XOR. Алгоритм состоит n повторения одной операции. Выбирается два блока данных одинакового размера, после чего производится вычисление значения формулы \ref{Feistel_equ}. В данной работе это пары соседних символов строки.
	\begin{equation}\label{Feistel_equ}
		(r \oplus ((l + st * 15))\:mod\:N_a,
	\end{equation}
	где 
	$l, r$ - выбраные блоки данных, 
	$st$ - номер итерации, 
	$N_a$ - размер алфавита строки.
	После чего данное значение записывается в $l$, а $l$ в $r$. После чего операция повторяется.

\subsection{Шифр Тритемиуса}
	Данный метод принято называть усложнённым шифром Цезаря. Ключевое отличие заключается в том, что в данном шифре помимо постоянной величины сдвига кодов символов могут использоваться и переменные значения. Например, позиция рассматриваемого символа или функция от этого значения.
	
	В данной работе используется формула осуществления сдвига \ref{shift_equ}
	\begin{equation}\label{shift_equ}
		str[i] = (pre[i] + floor(\sqrt{i}) - 3*i + 3)\:mod\:N_a,
	\end{equation}
	
	где 
	$i$ - позиция символа в строке, 
	$pre[i]$, $str[i]$ - исходное и зашифрованное значение i-го символа строки, 
	$N_a$ - размер алфавита строки.

\subsection{Шифр побитовыми сдвигами}
	Алгоритм использует в себе достаточно быстродейственную операцию циклического побитового сдвига на определённое число позиций. Сначала операция производится для набора байтов от 0-го до k-го, потом от 1-го до (k+1)-го и т.д. до конца строки. Такой метод позволяет достаточно существенно изменить исходный вид строки

\section{Конвейеризация алгоритма}
Идея конвейеризации алгоритма заключается в разделении задачи на стадии и распределение их между лентами конвейера. Каждой ленте выделяется один или более потоков, на которых последовательно отрабатываются одни и те же стадии разных задач. Сами задачи поступают на ленту из очереди для данного конвейера и после обработки лентой переходят в очередь следующей ленты или покидают конвейер.

В данной реализации каждая лента использует один поток и выполняет шифрование одним из вышеперечисленных алгоритмов. Очередь первой ленты заполняется до начала работы конвейера с помощью генератора случайных строк одинаковой длины. Обработанные задачи поступают в итоговую очередь, где будут обработаны статистические данные времён поступления и уходы с лент конвейера.

\section*{Вывод}
Результатом аналитического раздела стало определение цели и задач работы, описано понятие опереции шифрования, используемых алгоритмов, конвейеризации.