В ходе лабораторной работы достигнута поставленная цель: разработка и исследование конвейерного алгоритма шифрования строк. Решены все задачи работы.

Были изучены и описаны понятия конвейеризации и операции шифрования. Также были описан и реализован конвейерного алгоритма шифрования строк. Проведены замеры процессорного времени работы и простоя конвейера. На основании экспериментов проведён сравнительный анализ.

Из проведённых экспериментов было выявлено, что при использовании конвейерного алгоритма можно получить время обработки одной задачи сравнимое с непараллельным вариантом. Это выражено в результатах оценочного сравнения времени ожидания и общего времени обработки, а также сравнения с минимальными и максимальными величинами обработок и простоев. Значительного отличия минимального и максимального времени обработки отличаются незначительно, из чего можно сделать вывод о приемлемом размере очередей в момент пиковой загрузки.