\section{Цели и задачи работы}
Целью лабораторной работы является разработка и исследование параллельного алгоритма нахождения простых чисел.

Выделены следующие задачи лабораторной работы:

\begin{itemize}
\item описание понятия простого числа и параллелизма;
\item описание и реализация параллельного алгоритма нахождения множества простых чисел;
\item проведение замеров процессорного времени работы алгоритмов при различном количестве потоков;
\item анализ полученных результатов.
\end{itemize}

\section{Понятие простого числа}
Простое число --- натуральное число, которое делится на себя и на 1. При этом 1 простым числом не является\cite{prime_def}.

Стандартный алгоритм проверки числа $ N $ на простоту заключается в переборе всех чисел, которые могут быть делителем данного числа, то есть от 2 до $ \sqrt{N+1} $. В случае, если ни одно из этих чисел не делит $N$ без остатка, $N$ является простым числом.