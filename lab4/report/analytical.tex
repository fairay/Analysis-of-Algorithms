\section{Цели и задачи работы}
Целью лабораторной работы является оценка трудоёмкости алгоритмов сортировки.

Выделены следующие задачи лабораторной работы:

\begin{itemize}
\item описание операции сортировки;
\item описание и реализация алгоритмов сортировки;
\item проведение замеров процессорного времени работы алгоритмов при различных размерах массивов;
\item оценка трудоёмкости алгоритов;
\item проведение сравнительного анализа алгоритмов на основании экспериментов.
\end{itemize}

\section{Описание операции сортировки}
Сортировка массива по неубыванию - операция над массивом $arr[N]$, в результате которой в нём начинает выполняется условие\cite{sort_def}:
\begin{equation} 
	arr[i+1] >= arr[i], i \in [0, N-2]
\end{equation}

Аналогично формулируется определение для сортировки по невозрастанию. В случае, если в массиве нет равных элементов, также возможно применить операции сортироки по возрастанию и убыванию.

