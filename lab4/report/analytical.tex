\section{Цели и задачи работы}
Целью лабораторной работы является разработка и исследование параллельных алгоритмов умножения матриц.

Выделены следующие задачи лабораторной работы:

\begin{itemize}
\item описание понятия параллелизма и операции умножения матриц;
\item описание и реализация непараллельного и двух параллельных версий параллельного алгоритма умножения матриц;
\item проведение замеров процессорного времени работы алгоритмов при различном количестве потоков;
\item анализ полученных результатов.
\end{itemize}

\section{Математическое описание операции умножения матриц}
Умножение матриц - операция над матрицами А$[M*N] $ и B$[N*Q]$ \cite{mul_def}. Результатом операции является матрица C размерами $ M*Q $, в которой каждый элемент $c_{i,j}$ задаётся формулой
\begin{equation} 
	c_{i,j} = \sum_{k=1}^{N}(a_{i,k} \cdot b_{k,j})
\end{equation}


\section{Используемые алгоритмы}
Стандартный алгоритм подразумевает циклическое сложение всех элементов вышепредставленной суммы для получения каждого элемента матрицы $C$.

Параллелизм может быть досигнут за счёт выделения процессов, которые могут выполнятся независимо друг от друга. В данном случае вычисление каждого элемента $C$ ведётся независимо друг от друга, поэтому в качестве параллельных алгоритмов выбраны параллельное вычисление элементов строк и параллельное вычисление элементов столбцов.

\section*{Вывод}
Результатом аналитического раздела стало определение цели и задач работы, описано понятие опереции умножения матриц и описаны используемые алгоритмы.