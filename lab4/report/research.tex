\section{План экспериментов}
Измерения процессорного времени проводятся для поиска простых чисел до $10^{5}$. Изучается серия экспериментов с количеством потоков: $1, 2, 4, 8, 16, 32$

Для повышения точности, каждый замер производится пять раз, за результат берётся среднее арифметическое.

% //////////////
\section{Результат экспериментов}
По результатам измерений процессорного времени можно составить \hyperref[table_4_1]{таблицу 4.1}

\begin{table}[h] \label{table_4_1}
\caption{Результат измерений процессорного времени (в секундах)}
\begin{tabular}{| p{3.0cm} | c | c | c | c | c | c |}
	\hline
	Потоки			&$1$		&$2$		&$4$		&$8$		&$16$		&$32$		\\
	\hline\hline
	Время			&$0.033$	&$0.036$	&$0.013$	&$0.009$	&$0.012$	&$0.013$	\\
	\hline
\end{tabular}
\end{table}


Эксперименты проводились на компьютере с характеристиками:
\begin{itemize}
	\item ОС - Windows 10, 64 бит;
	\item Процессор -  Intel Core i7 8550U (1800 МГц, 4 ядра, 8 логических процессоров);
	\item Объем ОЗУ: 8 ГБ.
\end{itemize}

% //////////////
\section{Вывод}
По результатам экспериментов можно заключить следующее.
\begin{itemize}
	\item Наиболее быстродейственно алгоритм действует на 8 потоках, что равно количеству логических процессоров на испытуемом компьютере.
	\item Время при работе одного и двух потоков практически не отличаются. При этом, использование четырёх потоков даёт значительный выигрыш по времени.
	\item Использование 16 и 32 потоков показывает результат по времени хуже, чем при 8 потоках, из чего следует, что увеличение потоков даёт выигрыш по времени лишь до достижения количества логических ядер машины.
\end{itemize}


	