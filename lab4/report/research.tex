\section{План экспериментов}
Измерения процессорного времени проводятся на квадратных матрицах размером 1000х1000. Содержание матриц сгенерировано случайным образом. Изучается серия экспериментов с количеством потоков: $1, 2, 4, 8, 16, 32$

Для повышения точности, каждый замер производится пять раз, за результат берётся среднее арифметическое.

% //////////////
\section{Результат экспериментов}
По результатам измерений процессорного времени можно составить \hyperref[table_4_1]{таблицу 4.1}

\begin{table}[h] \label{table_4_1}
\caption{Результат измерений процессорного времени (в секундах)}
\begin{tabular}{| p{5.0cm} | l | c | c | c | c | c |}
	\hline
	Потоки						&$1$	&$2$	&$4$	&$8$	&$16$	&$32$	\\
	\hline\hline
	Многопоточно по строкам		&$8.22$	&$4.63$	&$2.64$	&$2.18$	&$2.22$	&$2.27$	\\
	\hline
	Многопоточно по столбцам	&$8.33$	&$5.64$	&$3.05$	&$2.62$	&$2.82$	&$2.89$	\\
	\hline
	Однопоточно					&\multicolumn{6}{|c|}{$7.13$}					\\
	\hline
\end{tabular}
\end{table}

Эксперименты проводились на компьютере с характеристиками:
\begin{itemize}
	\item ОС - Windows 10, 64 бит;
	\item Процессор -  Intel Core i7 8550U (1800 МГц, 4 ядра, 8 логических процессоров);
	\item Объем ОЗУ: 8 ГБ.
\end{itemize}

% //////////////
\section*{Вывод}
По результатам экспериментов можно заключить следующее.
\begin{itemize}
	\item Наиболее быстродейственно алгоритм действует на 8 потоках, что равно количеству логических процессоров на испытуемом компьютере.
	\item Использование по крайней мере двух потоков даёт ощутимый выигрыш по времени по сравнению с однопоточной версией алгоритма.
	\item Использование одного потока в многопоточных версиях алгоритма проигрывает по времени по сравнению с однопоточной версией алгоритма, что объясняется накладными расходами времени на управление потоками и mutex-ами.
	\item Параллельные версии алгоритма выполняются за приблизительно одинаковое время при одном потоке. Однако, использование большего количества потоков выявляет, что многопоточность по строкам в среднем на $18\%$ быстрее многопоточности по столбцам.
	\item Использование 16 и 32 потоков показывает результат по времени несколько хуже, чем при 8 потоках, из чего следует, что увеличение потоков даёт выигрыш по времени лишь до достижения количества логических ядер машины.
\end{itemize}


	