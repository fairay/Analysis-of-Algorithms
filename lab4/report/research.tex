\section{План экспериментов}
Измерения процессорного времени проводятся на квадратных матрицах с размерами: $32, 100, 250, 500, 1000$. Содержание матриц сгенерировано случайным образом. Изучается серия экспериментов с количеством потоков: $1, 2, 4, 8, 16, 32$

Для повышения точности, каждый замер производится пять раз, за результат берётся среднее арифметическое.

% //////////////
\section{Результат экспериментов}
По результатам измерений процессорного времени можно составить таблицы 
\hyperref[table_4_1]{4.1},
\hyperref[table_4_2]{4.2},
\hyperref[table_4_3]{4.3},
\hyperref[table_4_4]{4.4},
\hyperref[table_4_5]{4.5}

\begin{table}[h] \label{table_4_1}
\caption{Результат измерений процессорного времени, размер 32 (в миллисекундах)}
\begin{tabular}{| p{5.0cm} | c | c | c | c | c | c |}
	\hline
	Потоки						&$1$	&$2$	&$4$	&$8$	&$16$	&$32$	\\
	\hline\hline
	Многопоточно по строкам		&$0.24$	&$0.29$	&$0.36$	&$0.58$	&$1.04$	&$2.02$	\\
	\hline
	Многопоточно по столбцам	&$0.24$	&$0.30$	&$0.35$	&$0.59$	&$1.03$	&$1.97$	\\
	\hline
	Однопоточно					&\multicolumn{6}{|c|}{$0.096$}					\\
	\hline
\end{tabular}
\end{table}

\begin{table}[h] \label{table_4_2}
	\caption{Результат измерений процессорного времени, размер 100 (в миллисекундах)}
	\begin{tabular}{| p{5.0cm} | c | c | c | c | c | c |}
		\hline
		Потоки						&$1$	&$2$	&$4$	&$8$	&$16$	&$32$	\\
		\hline\hline
		Многопоточно по строкам		&$5.52$	&$2.73$	&$2.80$	&$2.78$	&$3.09$	&$3.51$	\\
		\hline
		Многопоточно по столбцам	&$4.87$	&$3.14$	&$2.87$	&$2.89$	&$3.15$	&$3.57$	\\
		\hline
		Однопоточно					&\multicolumn{6}{|c|}{$2.96$}					\\
		\hline
	\end{tabular}
\end{table}

\begin{table}[h] \label{table_4_3}
	\caption{Результат измерений процессорного времени, размер 250 (в секундах)}
	\begin{tabular}{| p{5.0cm} | c | c | c | c | c | c |}
		\hline
		Потоки						&$1$		&$2$		&$4$		&$8$		&$16$		&$32$	\\
		\hline\hline
		Многопоточно по строкам		&$0.068$	&$0.042$	&$0.031$	&$0.024$	&$0.026$	&$0.024$\\
		\hline
		Многопоточно по столбцам	&$0.064$	&$0.047$	&$0.035$	&$0.027$	&$0.034$	&$0.030$\\
		\hline
		Однопоточно					&\multicolumn{6}{|c|}{$0.051$}										\\
		\hline
	\end{tabular}
\end{table}

\begin{table}[h] \label{table_4_4}
	\caption{Результат измерений процессорного времени, размер 500 (в секундах)}
	\begin{tabular}{| p{5.0cm} | c | c | c | c | c | c |}
		\hline
		Потоки						&$1$	&$2$	&$4$	&$8$	&$16$	&$32$	\\
		\hline\hline
		Многопоточно по строкам		&$0.61$	&$0.38$	&$0.26$	&$0.20$	&$0.21$	&$0.20$\\
		\hline
		Многопоточно по столбцам	&$0.62$	&$0.42$	&$0.28$	&$0.23$	&$0.24$	&$0.24$\\
		\hline
		Однопоточно					&\multicolumn{6}{|c|}{$0.051$}										\\
		\hline
	\end{tabular}
\end{table}

\begin{table}[!h] \label{table_4_5}
	\caption{Результат измерений процессорного времени, размер 1000 (в секундах)}
	\begin{tabular}{| p{5.0cm} | c | c | c | c | c | c |}
		\hline
		Потоки						&$1$	&$2$	&$4$	&$8$	&$16$	&$32$	\\
		\hline\hline
		Многопоточно по строкам		&$8.22$	&$4.63$	&$2.64$	&$2.18$	&$2.22$	&$2.27$	\\
		\hline
		Многопоточно по столбцам	&$8.33$	&$5.64$	&$3.05$	&$2.62$	&$2.82$	&$2.89$	\\
		\hline
		Однопоточно					&\multicolumn{6}{|c|}{$7.13$}					\\
		\hline
	\end{tabular}
\end{table}

Эксперименты проводились на компьютере с характеристиками:
\begin{itemize}
	\item ОС - Windows 10, 64 бит;
	\item Процессор -  Intel Core i7 8550U (1800 МГц, 4 ядра, 8 логических процессоров);
	\item Объем ОЗУ: 8 ГБ.
\end{itemize}

% //////////////
\section*{Вывод}
По результатам экспериментов можно заключить следующее.
\begin{itemize}
	\item При относительно небольшом размере матриц (менее 100х100) использование потоков для уменьшения времени исполнения нецелесообразно, так как накладнымые расходы времени на управление потоками и mutex-ами больше, чем выигрыш от параллельного выполнения вычислений.
	\item Наиболее быстродейственно алгоритм действует на 8 потоках, что равно количеству логических процессоров на испытуемом компьютере.
	\item Использование по крайней мере двух потоков даёт ощутимый выигрыш по времени по сравнению с однопоточной версией алгоритма.
	\item Использование одного потока в многопоточных версиях алгоритма проигрывает по времени по сравнению с однопоточной версией алгоритма, что объясняется накладными расходами времени на управление потоками и mutex-ами.
	\item Параллельные версии алгоритма выполняются за приблизительно одинаковое время при одном потоке. Однако, использование большего количества потоков выявляет, что многопоточность по строкам быстрее многопоточности по столбцам вплоть до $20\%$.
	\item Использование 16 и 32 потоков показывает результат по времени несколько хуже, чем при 8 потоках, из чего следует, что увеличение потоков даёт выигрыш по времени лишь до достижения количества логических ядер машины.
\end{itemize}


	