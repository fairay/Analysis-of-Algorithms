\section{План экспериментов}
Измерения процессорного времени проводятся на массивах, заполненых целыми числами от $ -10^{4} $ до $ 10^{4} $. Содержание массивов сгенерировано случайным образом. Изучается серия экспериментов с размерностями массива: $20, 10^{2}, 10^{3}, 10^{4}, 10^{5}$

Для повышения точности, каждый замер производится пять раз, за результат берётся среднее арифметическое.

% //////////////
\section{Результат экспериментов}
По результатам измерений процессорного времени можно составить \hyperref[table_4_1]{таблицу 4.1}

\begin{table}[h] \label{table_4_1}
\caption{Результат измерений процессорного времени (в секундах)}
\begin{tabular}{| p{3.0cm} | c | c | c | c | c | c |}
	\hline
	Размерность		&$20$				&$10^{2}$			&$10^{3}$			&$10^{4}$			&$10^{5}$\\
	\hline\hline
	Пузырёк			&$1.3\cdot10^{-6}$	&$2.7\cdot10^{-5}$	&$2.0\cdot10^{-3}$	&$0.25$				&$30.7$	\\
	\hline
	Поразрядно		&$3.2\cdot10^{-6}$	&$1.1\cdot10^{-5}$	&$1.2\cdot10^{-4}$	&$1.1\cdot10^{-3}$	&$0.011$\\
	\hline
	Слиянием		&$8.1\cdot10^{-7}$	&$6.8\cdot10^{-6}$	&$8.0\cdot10^{-5}$	&$1.2\cdot10^{-3}$	&$0.014$\\
	\hline
\end{tabular}
\end{table}


Эксперименты проводились на компьютере с характеристиками:
\begin{itemize}
	\item ОС - Windows 10, 64 бит;
	\item Процессор -  Intel Core i7 8550U (1800 МГц, 4 ядра, 8 логических процессоров);
	\item Объем ОЗУ: 8 ГБ.
\end{itemize}

% //////////////
\section{Вывод}
По результатам экспериментов можно заключить следующее.
\begin{itemize}
	\item Сортировка пузырьком показывает самый большой прирост по процессорному времени по сравнению с другими алгоритмами при увеличении размерности массивов.
	\item Поразрядная сортировка уступает в скорости сортирвке слиянием при относительно небольших размерах массивов. При больших размерах поразрядная сортировка показывет себя лучше, но крайне незначительно. При этом, сортировка слиянием способна работать с большим типом данных.
	\item При увеличении размера массивов в 10 раз, наблюдается рост затраченного процессорного времени для сортировки пузырьком примерно в 100 раз, для поразрядной в 10 раз и для сортировки слиянием в 12 раз, что в целом соответсвует расчётам их трудоёмкости.
\end{itemize}


	