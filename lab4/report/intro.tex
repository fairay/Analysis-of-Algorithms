В данной лабораторной реализуется и оценивается параллельный алгоритм стандартного умножения матриц.

Параллелизм --- выполнение нескольких вычислений в различных потоках. Параллельное программирование интересно тем, что в процессоре с многоядерной архитектурой несколько процессов могут выполнятся одновременно на разных ядрах. Это приводит к тому, что время выполнения параллельного алгоритма может быть ощутимо меньше, чем у его однопоточного аналога.

Стандартный алгоритм умножения матриц подразумевает проход по всем элементам результирующей матрицы $C$ для вычисления их значений. Так как вычисление каждого элемента независимо, этот алгоритм подходит для реализации покоординатного параллелизма.