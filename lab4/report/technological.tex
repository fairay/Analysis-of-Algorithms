\section{Выбор языка программирования}
В качестве языка программирования был выбран C++\cite{C++_Doc}, так как имеется опыт работы с ним, и с библиотеками, позволяющими провести исследование и тестирование программы. Также в языке имеются средства для использования многопоточности. Разработка проводилась в среде Visual Studio 2019\cite{VisualStudio}.


\section{Листинг кода}
Реализация алгоритмов умножения матриц представлена на листингах 3.1-3.3

\begin{lstlisting}[caption = {Функция однопоточного умножения матриц}, captionpos=b]
mat_t std_mult(mat_t a, mat_t b, int m, int n, int q, int)
{
	mat_t c = create_mat(m, q);
	for (int i = 0; i < m; i++)
		for (int j = 0; j < q; j++)
		{
			double res = 0;
			for (int k = 0; k < n; k++)
				res += a[i][k] * b[k][j];
			c[i][j] = res;
		}
	return c;
}
\end{lstlisting}

\begin{lstlisting}[caption = {Функция умножения матриц (параллельная по строкам)}, captionpos=b]
using namespace std;
std::mutex mtx;

void _mult_thread(mat_t& c, mat_t& a, mat_t& b, int m, int n, int q, int start, int step)
{
	for (int i = start; i < m; i+= step)
		for (int j = 0; j < q; j++)
		{
			double res = 0;
			for (int k = 0; k < n; k++)
				res += a[i][k] * b[k][j];
			mtx.lock();
			c[i][j] = res;
			mtx.unlock();
		}
}

mat_t std_mult_thread1(mat_t a, mat_t b, int m, int n, int q, int thread_n)
{
	mat_t c = create_mat(m, q);
	
	vector<thread> thread_arr;
	for (int i = 0; i < thread_n; i++)
		thread_arr.push_back(thread(_mult_thread, ref(c), ref(a), ref(b), m, n, q, i, thread_n));
	
	for (int i = 0; i < thread_n; i++)
		thread_arr[i].join();
	
	return c;
}
\end{lstlisting}

\begin{lstlisting}[caption = {Функция умножения матриц (параллельная по столбцам)}, captionpos=b]
using namespace std;
std::mutex mtx;

void _mult_thread2(mat_t& c, mat_t& a, mat_t& b, int m, int n, int q, int start, int step)
{
	for (int i = 0; i < m; i++)
		for (int j = start; j < q; j+=step)
		{
			double res = 0;
			for (int k = 0; k < n; k++)
				res += a[i][k] * b[k][j];
			mtx.lock();
			c[i][j] = res;
			mtx.unlock();
		}
}

mat_t std_mult_thread2(mat_t a, mat_t b, int m, int n, int q, int thread_n)
{
	mat_t c = create_mat(m, q);
	
	vector<thread> thread_arr;
	for (int i = 0; i < thread_n; i++)
		thread_arr.push_back(thread(_mult_thread2, ref(c), ref(a), ref(b), m, n, q, i, thread_n));
	
	for (int i = 0; i < thread_n; i++)
		thread_arr[i].join();
	
	return c;
}
\end{lstlisting}


\section{Результаты тестирования}
Для тестирования написанных функций был создан отдельный файл с ранее описанными классами тестов. Тестирование функций проводилось за счёт сравнения результов функций друг с другом.

Состав тестов приведён в листинге 3.4.

\begin{lstlisting}[caption = {Модульные тесты}, captionpos=b]
#include "tests.h"

using namespace std;

// Сравнение работы функций
bool _cmp_funcs(mat_t a, mat_t b, int m, int n, int q, int thread_n)
{
	mat_t c0 = std_mult(a, b, m, n, q, thread_n);
	mat_t c1 = std_mult_thread1(a, b, m, n, q, thread_n);
	mat_t c2 = std_mult_thread2(a, b, m, n, q, thread_n);
	
	bool flag = cmp_matrix(c0, c1, m, q);
	if (flag)
		flag = cmp_matrix(c1, c2, m, q);
		
	free_mat(&c0, m, q);
	free_mat(&c1, m, q);
	free_mat(&c2, m, q);
	return flag;
}

// Размер матриц = 1
void _test_size_one()
{
	mat_t a = create_mat(1, 1);
	mat_t b = create_mat(1, 1);
	
	a[0][0] = 0;
	b[0][0] = 1;
	if (!_cmp_funcs(a, b, 1, 1, 1, 1))
	{
		std::cout << __FUNCTION__ << " - FAILED\n";
		return;
	}
	
	a[0][0] = 3;
	b[0][0] = 4;
	if (!_cmp_funcs(a, b, 1, 1, 1, 16))
	{
		std::cout << __FUNCTION__ << " - FAILED\n";
		return;
	}
	
	free_mat(&a, 1, 1);
	free_mat(&b, 1, 1);
	
	std::cout << __FUNCTION__ << " - OK\n";
}

// Пустые матрицы
void _test_void()
{
	mat_t a = random_matrix(3, 2);
	mat_t b = void_matrix(2, 1);
	if (!_cmp_funcs(a, b, 3, 2, 1, 1))
	{
		std::cout << __FUNCTION__ << " - FAILED\n";
		return;
	}
	free_mat(&a, 3, 2);
	
	a = void_matrix(3, 2);
	if (!_cmp_funcs(a, b, 3, 2, 1, 1))
	{
		std::cout << __FUNCTION__ << " - FAILED\n";
		return;
	}
	free_mat(&a, 3, 2);
	free_mat(&b, 2, 1);
	std::cout << __FUNCTION__ << " - OK\n";
}

// Сравнение работы при различном количестве потоков
void _test_threads()
{
	mat_t a = random_matrix(50, 50);
	mat_t b = random_matrix(50, 50);
	
	for (int i = 1; i <= 16; i++)
	{
		if (!_cmp_funcs(a, b, 50, 50, 50, i))
		{
			std::cout << __FUNCTION__ << " - FAILED\n";
			break;
		}
	}
	
	free_mat(&a, 50, 50);
	free_mat(&b, 50, 50);
	
	std::cout << __FUNCTION__ << " - OK\n";
}

void run_tests()
{
	_test_size_one();
	_test_void();
	_test_threads();
}
\end{lstlisting}

Все тесты пройдены успешно.

\section{Оценка времени}
Для замера процессорного времени исполнения функции используется функция QueryPerformanceCounter библиотеки windows.h\cite{QueryPerformanceCounter}. Проведение измерений производится в функциях, приведённых в листинге 3.3.

\begin{lstlisting}[caption = {Функции замера процессорного времени работы функции}, captionpos=b]
void test_time(mult_f f, int thread_n)
{
	int n = 1000;
	mat_t a = random_matrix(n, n);
	mat_t b = random_matrix(n, n);
	mat_t c;
	
	int count = 0;
	start_counter();
	while (get_counter() < 3.0 * 1000) 
	{
		c = f(a, b, n, n, n, thread_n);
		free_mat(&c, n, n);
		count++;
	}
	double t = get_counter() / 1000;
	
	start_counter();
	for (int i = 0; i < count; i++)
	{
		c = create_mat(n, n);
		free_mat(&c, n, n);
	}
	t -= get_counter() / 1000;
	
	cout << "Выполнено " << count << " операций за " << t << " секунд" << endl;
	cout << "Время: " << t / count << endl;
	
	free_mat(&a, n, n);
	free_mat(&b, n, n);
}

void experiments_series(vector<int>& a)
{
	for (int i : a)
	{
		cout << "=======================================" << endl;
		cout << "\nКоличество потоков: " << i << endl;
		
		cout << "Стандартное умножение(многопоточно по строкам)" << endl;
		test_time(std_mult_thread1, i);
		cout << "Стандартное умножение(многопоточно по столбцам)" << endl;
		test_time(std_mult_thread2, i);
		
		cout << "\nСтандартное умножение(многопоточно по строкам)" << endl;
		test_time(std_mult_thread1, i);
		cout << "Стандартное умножение(многопоточно по столбцам)" << endl;
		test_time(std_mult_thread2, i);
		
		cout << "=======================================" << endl;
	}
	cout << "Стандартное умножение (однопоточно)" << endl;
	test_time(std_mult, 1);
}

\end{lstlisting}

\section*{Вывод}
Результатом технологической части стал выбор используемых технических средств реализации и реализация алгоритмов, системы тестов и замера времени работы на языке С++.
