\section{Выбор языка программирования}
В качестве языка программирования был выбран C++\cite{C++_Doc}, так как имеется опыт работы с ним, и с библиотеками, позволяющими провести исследование и тестирование программы. Также в языке имеются средства для использования многопоточности. Разработка проводилась в среде Visual Studio 2019\cite{VisualStudio}.


\section{Листинг кода}
Реализация алгоритма поиска простых чисел представлена на листинге 3.1

\begin{lstlisting}[caption = Функция параллельного поиска простых чисел.]
#include "primes.h"

using namespace std;
std::mutex set_mtx;

inline bool is_prime_n(n_t n)
{
	double max_check = sqrt(n + 1);
	for (n_t i = 2; i < max_check; i++)
	if (n % i == 0)
	return false;
	return true;
}

void thread_proc(set<n_t>& set, n_t start_n, n_t end_n, int step)
{
	for (n_t i = start_n; i <= end_n; i += step)
	if (is_prime_n(i))
	{
		set_mtx.lock();
		set.insert(i);
		set_mtx.unlock();
	}
}

set<n_t> find_primes(n_t max_n, int thread_n)
{
	set<n_t> set;
	vector<thread> thread_arr;
	
	n_t start_n;
	for (int i = 0; i < thread_n; i++)
	{
		start_n = static_cast<n_t>(i) + 2;
		thread_arr.push_back(thread(thread_proc, ref(set), start_n, max_n, thread_n));
	}
	
	for (int i = 0; i < thread_n; i++)
	thread_arr[i].join();
	
	return set;
}
\end{lstlisting}


\section{Результаты тестирования}
Для тестирования написанных функций был создан отдельный файл с ранее описанными классами тестов. Тестирование функций проводилось за счёт сравнения результов функций друг с другом (в случае разного количества потоков) и заданным ожидаемым результатом.

Состав тестов приведён в листинге 3.2.

\begin{lstlisting}[caption = Модульные тесты]
#include "tests.h"

using namespace std;

// Сравнение множеств
bool cmp_sets(vector<set<n_t>>& all_results)
{
	int res_size = all_results.size();
	for (int i = 0; i < res_size -1; i++)
	if (all_results[i].size() != all_results[i + 1].size())
	return false;
	
	for (auto n : all_results[0])
	for (int i = 1; i < res_size; i++)
	all_results[i].erase(n);
	all_results[0].clear();
	
	for (int i = 0; i < res_size-1; i++)
	if (all_results[i].size() != all_results[i + 1].size())
	return false;
	return true;
}

// Сравнение разультата работы при разном количестве потоков
void _test_threads(void)
{
	std::string msg;
	msg = __FUNCTION__;		msg += " - OK";
	
	n_t max_n = 1000;
	vector<set<n_t>> all_results;
	
	for (int i = 0; i < 20; i++)
	all_results.push_back(find_primes(max_n, i+1));
	
	if (!cmp_sets(all_results)) 
	{
		msg = __FUNCTION__;		msg += " - FAILED";
	}
	
	std::cout << msg << std::endl;
}

// Тест при максимальном числе = 2
void _test_2(void)
{
	std::string msg;
	msg = __FUNCTION__;		msg += " - OK";
	
	vector<set<n_t>> all_results;
	
	all_results.push_back(set<n_t>{2});
	all_results.push_back(find_primes(2, 1));
	
	if (!cmp_sets(all_results))
	{
		msg = __FUNCTION__;		msg += " - FAILED";
	}
	
	std::cout << msg << std::endl;
}

// Тест при максимальном числе = 200
void _test_200(void)
{
	std::string msg;
	msg = __FUNCTION__;		msg += " - OK";
	
	vector<set<n_t>> all_results;
	
	all_results.push_back(set<n_t>{2, 3, 5, 7, 11, 13, 17, 19, 23,
		29, 31, 37, 41, 43, 47, 53, 59, 61, 67, 71, 73, 79, 83, 89,
		97, 101, 103, 107, 109, 113, 127, 131, 137, 139, 149, 151,
		157, 163, 167, 173, 179, 181, 191, 193, 197, 199});
	all_results.push_back(find_primes(200, 1));
	
	if (!cmp_sets(all_results))
	{
		msg = __FUNCTION__;		msg += " - FAILED";
	}
	
	std::cout << msg << std::endl;
}

void run_tests()
{
	_test_threads();
	_test_2();
	_test_200();
}
\end{lstlisting}

Все тесты пройдены успешно.

\section{Оценка времени}
Для замера процессорного времени исполнения функции используется функция QueryPerformanceCounter библиотеки windows.h\cite{QueryPerformanceCounter}. Проведение измерений производится в функциях, приведённых в листинге 3.3.

\begin{lstlisting}[caption = Функции замера процессорного времени работы функции]
void test_time(n_t max_n, int thread_n)
{
	int count = 0;
	start_counter();
	while (get_counter() < 3.0 * 1000)
	{
		find_primes(max_n, thread_n);
		count++;
	}
	double t = get_counter() / 1000;
	cout << "Выполнено " << count << " операций за " << t << " секунд" << endl;
	cout << "Время: " << t / count << endl;
}

void experiments_series(n_t max_n, vector<int>& thread_arr)
{
	cout << "Размер: " << max_n << endl;
	for (int i : thread_arr)
	{
		cout << "\nЧисло потоков = " << i << endl;
		test_time(max_n, i);
	}
}
\end{lstlisting}

\section*{Вывод}
Результатом технологической части стал выбор используемых технических средств реализации и реализация алгоритма, системы тестов и замера времени работы на языке С++.
