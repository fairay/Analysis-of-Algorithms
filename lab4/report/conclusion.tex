В ходе лабораторной работы достигнута поставленная цель: разработка и исследование параллельных алгоритмов умножения матриц. Решены все задачи работы.

Были изучены и описаны понятия параллелизма и операции умножения матриц. Также были описан и реализованы непараллельный и две параллельные версии параллельного алгоритма умножения матриц. Проведены замеры процессорного времени работы алгоритмов при различном количестве потоков. На основании экспериментов проведён сравнительный анализ.

Из проведённых экспериментов было выявлено, что наиболее быстродейственным является использование количества потоков, которое равно количеству логических ядер компьютера. Увеличение или уменьшение количества потоков ведёт к большему времени выполнения вычислений. Однако, использование потоков даёт выигрыш по времени работы только для относительно больших размеров матриц, иначе их использование лишь увеличит время вычислений за счёт накладных расходов.Также было установлено, что алгоритм, использующий многопоточность по строкам показывает себя несколько быстрее алгоритма с многопоточностью по столбцам.